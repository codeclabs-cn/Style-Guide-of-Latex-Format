%! Author = codeclabs-cn
%! Date = 2023/1/3

\chapter{C++ Generated Code}\label{ch:c++-generated-code}
This page describes exactly what C++ code the protocol buffer compiler generates for any given protocol definition. Any differences between proto2 and proto3 generated code are highlighted - note that these differences are in the generated code as described in this document, not the base message classes/interfaces, which are the same in both versions. You should read the proto2 language guide and/or proto3 language guide before reading this document.

\section{Compiler Invocation}\label{sec:compiler-invocation}
The protocol buffer compiler produces C++ output when invoked with the \mintinline{Protobuf}{--cpp_out=} command-line flag. The parameter to the \mintinline{Protobuf}{--cpp_out=} option is the directory where you want the compiler to write your C++ output. The compiler creates a header file and an implementation file for each \mintinline{Protobuf}{-.proto} file input. The names of the output files are computed by taking the name of the \mintinline{Protobuf}{-.proto} file and making two changes:
So, for example, let's say you invoke the compiler as follows:
\begin{minted}{Protobuf}
    protoc --proto_path=src --cpp_out=build/gen src/foo.proto src/bar/baz.proto
\end{minted}

The compiler will read the files \mintinline{Protobuf}{src/foo.proto} and \mintinline{Protobuf}{src/bar/baz.proto} and produce four output files: \mintinline{Protobuf}{build/gen/foo.pb.h}, \mintinline{Protobuf}{build/gen/foo.pb.cc}, \mintinline{Protobuf}{build/gen/bar/baz.pb.h}, \mintinline{Protobuf}{build/gen/bar/baz.pb.cc}. The compiler will automatically create the directory \mintinline{Protobuf}{build/gen/bar} if necessary, but it will not create build or \mintinline{Protobuf}{build/gen}; they must already exist.

\section{Packages}\label{sec:packages}
If a \mintinline{Protobuf}{.proto} file contains a \mintinline{Protobuf}{package} declaration, the entire contents of the file will be placed in a corresponding C++ namespace. For example, given the \mintinline{Protobuf}{package} declaration:
\begin{minted}{Protobuf}
package foo.bar;
\end{minted}
All declarations in the file will reside in the \mintinline{Protobuf}{foo::bar} namespace.

\section{Messages}
Given a simple message declaration:
\begin{minted}{Protobuf}
    message Foo {}
\end{minted}
The protocol buffer compiler generates a class called \mintinline{Protobuf}{Foo}, which publicly derives from google::protobuf::Message. The class is a concrete class; no pure-virtual methods are left unimplemented. Methods that are virtual in Message but not pure-virtual may or may not be overridden by \mintinline{Protobuf}{Foo}, depending on the optimization mode. By default, \mintinline{Protobuf}{Foo} implements specialized versions of all methods for maximum speed. However, if the \mintinline{Protobuf}{.proto} file contains the line:
\begin{minted}{Protobuf}
    option optimize_for = CODE_SIZE;
\end{minted}
then \mintinline{Protobuf}{Foo} will override only the minimum set of methods necessary to function and rely on reflection-based implementations of the rest. This significantly reduces the size of the generated code, but also reduces performance. Alternatively, if the \mintinline{Protobuf}{.proto} file contains:
\begin{minted}{Protobuf}
    option optimize_for = LITE_RUNTIME;
\end{minted}

then \mintinline{Protobuf}{Foo} will include fast implementations of all methods, but will implement the google::protobuf::MessageLite interface, which only contains a subset of the methods of \mintinline{Protobuf}{Message}. In particular, it does not support descriptors or reflection. However, in this mode, the generated code only needs to link against \mintinline{Protobuf}{libprotobuf-lite.so} (\mintinline{Protobuf}{libprotobuf-lite.lib} on Windows) instead of \mintinline{Protobuf}{libprotobuf.so} (\mintinline{Protobuf}{libprotobuf.lib}). The \enquote{lite} library is much smaller than the full library, and is more appropriate for resource-constrained systems such as mobile phones.

You should \emph{not} create your own \mintinline{Protobuf}{Foo} subclasses. If you subclass this class and override a virtual method, the override may be ignored, as many generated method calls are de-virtualized to improve performance.

The \mintinline{Protobuf}{Message} interface defines methods that let you check, manipulate, read, or write the entire message, including parsing from and serializing to binary strings.
\begin{itemize}
    \item \mintinline{C++}{bool ParseFromString(const string& data)}: Parse the message from the given serialized binary string (also known as wire format).
    \item \mintinline{C++}{bool SerializeToString(string* output) const}: Serialize the given message to a binary string.
    \item \mintinline{C++}{string DebugString()}: Return a string giving the `text\_format` representation of the proto (should only be used for debugging).
\end{itemize}

In addition to these methods, the \mintinline{C++}{Foo} class defines the following methods:
\begin{itemize}
    \item \mintinline{C++}{Foo()}: Default constructor.
    \item \mintinline{C++}{~Foo()}: Default destructor.
    \item \mintinline{C++}{Foo(const Foo& other)}: Copy constructor.
    \item \mintinline{C++}{Foo(Foo&& other)}: Move constructor.
    \item \mintinline{C++}{Foo& operator=(const Foo& other)}: Assignment operator.
    \item \mintinline{C++}{Foo& operator=(Foo&& other)}: Move-assignment operator.
    \item \mintinline{C++}{void Swap(Foo* other)}: Swap content with another message.
    \item \mintinline{C++}{const UnknownFieldSet& unknown_fields() const}: Returns the set of unknown fields encountered while parsing this message.
    \item \mintinline{C++}{UnknownFieldSet* mutable_unknown_fields()}: Returns a pointer to the mutable set of unknown fields encountered while parsing this message.
\end{itemize}

The class also defines the following static methods:
\begin{itemize}
    \item \mintinline{C++}{static const Descriptor* descriptor()}: Returns the type's descriptor. This contains information about the type, including what fields it has and what their types are. This can be used with reflection to inspect fields programmatically.
    \item \mintinline{C++}{static const Foo& default_instance()}: Returns a const singleton instance of Foo which is identical to a newly-constructed instance of Foo (so all singular fields are unset and all repeated fields are empty). Note that the default instance of a message can be used as a factory by calling its New() method.
\end{itemize}









