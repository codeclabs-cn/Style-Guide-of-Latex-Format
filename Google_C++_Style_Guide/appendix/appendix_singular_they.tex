%! Author = codeclabs-cn
%! Date = 2023/1/3

\chapter{Singular \enquote{They}}\label{ch:singular-they}

The singular \enquote{they} is a generic third-person singular pronoun in English. Use of the singular \enquote{they} is endorsed as part of APA Style because it is inclusive of all people and helps writers avoid making assumptions about gender. Although usage of the singular \enquote{they} was once discouraged in academic writing, many advocacy groups and publishers have accepted and endorsed it, including Merriam-Webster’s Dictionary.

Always use a person’s self-identified pronoun, including when a person uses the singular \enquote{they} as their pronoun.Also use \enquote{they} as a generic third-person singular pronoun to refer to a person whose gender is unknown or irrelevant to the context of the usage.Do not use \enquote{he} or \enquote{she} alone as generic third-person singular pronouns. Use combination forms such as \enquote{he or she} and \enquote{she or he} only if you know that these pronouns match the people being described.Do not use combination forms such as \enquote{(s)he} and \enquote{s/he}. If you do not know the pronouns of the person being described, reword the sentence to avoid a pronoun or use the pronoun \enquote{they}.

\section{Forms of the singular \enquote{they}}
Use following forms of the singular \enquote{they}:

    \begin{table}[htb]
        \centering
        \begin{tabular}{|c|l|}
            \hline
            \rowcolor[HTML]{DDE7FA}
            Form                     & Example                                                              \\ \hline
            they                     & Casey is a gender-fluid person. They are from Texas and enjoy tacos. \\ \hline
            them                     & Every client got a care package delivered to them.                   \\ \hline
            their                    & Each child played with their parent.                                 \\ \hline
            theirs                   & The cup of coffee is theirs.                                         \\ \hline
            themselves (or themself) & A private person usually keeps to themselves {[}or themself{]}.      \\ \hline
        \end{tabular}
    \end{table}

Here are some tips to help you use the proper forms:
\begin{itemize}
    \item Use a plural verb form with the singular pronoun \enquote{they} (i.e., write \enquote{they are} not \enquote{they is}).
    \item Use a singular verb form with a singular noun (i.e., write \enquote{Casey is} or \enquote{a person is}, not \enquote{Casey are} or \enquote{a person are}).
    \item Both \enquote{themselves} and \enquote{themself} are acceptable as reflexive singular pronouns; however, \enquote{themselves} is currently the more common usage.
\end{itemize}

\section{Alternatives to the generic singular \enquote{they}}

If using the singular \enquote{they} as a generic third-person pronoun seems awkward, try rewording the sentence or using the plural.

    \begin{table}[htb]
        \centering
        \begin{tabular}{|c|l|}
            \hline
            \rowcolor[HTML]{DDE7FA}
            Strategy               & Example                                    \\ \hline
            Rewording the sentence & I delivered a care package to the client.  \\ \hline
            Using the plural       & Private people usually keep to themselves. \\ \hline
        \end{tabular}
    \end{table}

However, do not use alternatives when people use \enquote{they} as their pronoun—always use the pronouns that people use to refer to themselves.