%! Author = codeclabs-cn
%! Date = 2023/1/4

\chapter{Tip of the Week \#109: Meaningful \texorpdfstring{\mintinline{C++}{const}}{} in Function Declarations}\label{ch:tip-of-the-week-109}
\chapterauthor{Greg Miller (jgm@google.com)}

This document will explain when \mintinline{C++}{const} is meaningful in function declarations, and when it is meaningless and best omitted. But first, let us briefly explain what is meant by the terms \emph{declaration} and \emph{definition}.

Consider the following code:
\begin{minted}{C++}
void F(int);                     // 1: declaration of F(int)
void F(const int);               // 2: re-declaration of F(int)
void F(int) { /* ... */ }        // 3: definition of F(int)
void F(const int) { /* ... */ }  // 4: error: re-definition of F(int)
\end{minted}

The first two lines are function \emph{declaration}s. A function \emph{declaration} tells the compiler the function’s signature and return type. In the above example, the function’s signature is \mintinline{C++}{F(int)}. The constness of the function’s parameter type is ignored, so both declarations are equivalent (See \href{http://eel.is/c++draft/over.load}{Overloadable declarations}.)

Lines 3 and 4 from the above code are both function \emph{definition}s. A function \emph{definition} is also a declaration, but a definition also contains the function’s body. Therefore, line 3 is a definition for a function with the signature \mintinline{C++}{F(int)}. Similarly, line 4 is also a definition for the same function, \mintinline{C++}{F(int)}, which will result in an error at link time. Multiple declarations are allowed, but only a single definition is permitted.

Even though the definitions on lines 3 and 4 \emph{declare} and \emph{define} the same function, there is a difference within their function bodies due to the way they are declared. From the definition on line 3, the type of the function-parameter variable within the function will be \mintinline{C++}{int} (i.e., non-const). On the other hand, the definition on line 4 will produce a function-parameter variable within the function whose type is \mintinline{C++}{const int}.

\section{Meaningful \texorpdfstring{\mintinline{C++}{const}}{} in Function Declarations}
Not all \mintinline{C++}{const} qualifications in function declarations are ignored. To quote from \href{http://eel.is/c++draft/over.load}{Overloadable declarations} ([over.load]) in the C++ standard (emphasis added):

\say{const \textit{type-specifiers \textbf{buried within a parameter type specification} are significant and can be used to distinguish overloaded function declarations}}

The following are examples where \mintinline{C++}{const} is significant and not ignored:
\begin{minted}{C++}
void F(const int* x);                  // 1
void F(const int& x);                  // 2
void F(std::unique_ptr<const int> x);  // 3
void F(int* x);                        // 4
\end{minted}

In the above examples, the \mintinline{C++}{x} parameter itself is never declared \mintinline{C++}{const}. Each of the above functions accepts a parameter named \mintinline{C++}{x} of a different type, thus forming a valid overload set. Line 1 declares a function that accepts a \enquote{pointer to an \mintinline{C++}{int} that is \mintinline{C++}{const}}. Line 2 declares a function that accepts a \enquote{reference to an \mintinline{C++}{int} that is \mintinline{C++}{const}}. And line 3 declares a function that accepts a \enquote{\mintinline{C++}{std::unique_ptr} to an \mintinline{C++}{int} that is \mintinline{C++}{const}}. All of these uses of \mintinline{C++}{const} are important and not ignored because they are part of the parameter type specification and are not top-level \mintinline{C++}{const} qualifications that affect the parameter \mintinline{C++}{x} itself.

Line 4 is interesting because it does not include the \mintinline{C++}{const} keyword at all, and may at first appear to be equivalent to the declaration on line 1 given the reasons cited at the beginning of this document. The reason that this is not true and that line 4 is a valid and distinct declaration is that only top-level, or outermost, \mintinline{C++}{const} qualifications of the parameter type specification are ignored.

To complete this example, let us look at a few more examples where a const is meaningless and ignored.
\begin{minted}{C++}
void F(const int x);          // 1: declares F(int)
void F(int* const x);         // 2: declares F(int*)
void F(const int* const x);   // 3: declares F(const int*)
\end{minted}

\section{Rules of Thumb}
Though few of us will ever truly master all the delightful obscurities of C++, it is important that we do our best to understand the rules of the game. This will help us write code that is understood by other C++ programmers who are following the same rules and playing the same game. For this reason, it is important that we understand when \mintinline{C++}{const} qualification is meaningful in a function declaration and when it is ignored.

Although there is no official guidance from the Google C++ style guide, and there is no single generally accepted opinion, the following is one reasonable set of guidelines:
\begin{itemize}
    \item Never use top-level \mintinline{C++}{const} on function parameters in \emph{declaration}s that are not definitions (and be careful not to copy/paste a meaningless \mintinline{C++}{const}). It is meaningless and ignored by the compiler, it is visual noise, and it could mislead readers.
    \item Do use top-level \mintinline{C++}{const} on function parameters in \emph{definition}s at your (or your team’s) discretion. You might follow the same rationale as you would for when to declare a function-local variable \mintinline{C++}{const}.
\end{itemize}


