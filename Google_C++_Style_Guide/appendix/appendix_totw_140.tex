%! Author = codeclabs-cn
%! Date = 2023/1/3

\chapter{Tip of the Week \#140: Constants: Safe Idioms}\label{ch:tip-of-the-week-140}
\chapterauthor{Matt Armstrong}

What are best ways to express constants in C++? You probably have an idea of what the word means in English, but it is easy to express the concept incorrectly in code. Here we’ll first define a few key concepts, then get to a list of safe techniques. For the curious, we then go into more detail about what can go wrong, and describe a C++17 language feature that makes expressing constants easier.

There is no formal definition of a \enquote{C++ constant} so let’s propose an informal one.
\begin{itemize}
    \item \textbf{The value}: A value never changes; five is always five. When we want to express a constant, we need a value, but only one.
    \item \textbf{The object}: At each point in time an object has a value. C++ places a strong emphasis on mutable objects, but mutation of constants is disallowed.
    \item \textbf{The name}: Named constants are more useful than bare literal constants. Both variables and functions can evaluate to constant objects.
\end{itemize}
Putting that all together, let’s call a constant a variable or function that always evaluates to the same value. There are a few more key concepts.
\begin{itemize}
    \item \textbf{Safe Initialization}: Many times constants are expressed as values in static storage, which must be safely initialized. For more on that, see \hyperref[sec:static-and-global-variables]{the C++ Style Guide}.
    \item \textbf{Linkage}: Linkage has to do with how many instances (or \enquote{copies}) of a named object there are in a program. It is usually best for a constant with one name to refer to a single object within the program. For global or namespace-scoped variables this requires something called external linkage (\href{http://en.cppreference.com/w/cpp/language/storage_duration}{you can read more about linkage here}).
    \item \textbf{Compile-time evaluation}: Sometimes the compiler can do a better job optimizing code if a constant’s value is known at compile time. This benefit can sometimes justify defining the values of constants in header files, despite the additional complexity.
\end{itemize}

When we say we’re \enquote{adding a constant} we’re actually declaring an API and defining its implementation in such a way that satisfies most or all of the above criteria. The language doesn’t dictate how we do this, and some ways are better than others. Often the simplest approach is declaring a \mintinline{C++}{const} or \mintinline{C++}{constexpr} variable, marked as \mintinline{C++}{inline} if it’s in a header file. Another approach is returning a value from a function, which is more flexible. We’ll cover examples of both approaches.

A note on \mintinline{C++}{const}: it isn’t enough. A \mintinline{C++}{const} object is read-only but this does not imply that it is \emph{immutable} nor does it imply that the value is always the same. The language provides ways to mutate values we think of as \mintinline{C++}{const}, such as the \mintinline{C++}{mutable} keyword and \mintinline{C++}{const_cast}. But even straightforward code can demonstrate the point:
\begin{minted}{C++}
void f(const std::string& s) {
  const int size = s.size();
  std::cout << size << '\n';
}

f("");  // Prints 0
f("foo");  // Prints 3
\end{minted}
In the above code \mintinline{C++}{size} is a \mintinline{C++}{const} variable, yet it holds multiple values as the program runs. It is not a constant.

\section{Constants in Header Files}
All of the idioms in this section are robust and recommendable.
\subsubsection{An inline constexpr Variable}
From C++17 variables can be marked as \mintinline{C++}{inline}, ensuring that there is only a single copy of the variable. When used with \mintinline{C++}{constexpr} to ensure safe initialization and destruction this gives another way to define a constant whose value is accessible at compile time.
\begin{minted}{C++}
// in foo.h
inline constexpr int kMyNumber = 42;
inline constexpr absl::string_view kMyString = "Hello";
\end{minted}

\subsection{An extern const Variable}\label{subsec:an-extern-const-variable}
\begin{minted}{C++}
// Declared in foo.h
ABSL_CONST_INIT extern const int kMyNumber;
ABSL_CONST_INIT extern const char kMyString[];
ABSL_CONST_INIT extern const absl::string_view kMyStringView;
\end{minted}
The above example \textbf{declares} \emph{one} instance of each object. The \mintinline{C++}{extern} keyword ensures external linkage. The \mintinline{C++}{const} keyword helps prevent accidental mutation of the value. This is a fine way to go, though it does mean the compiler can’t \enquote{see} the constant values. This limits their utility somewhat, but not in ways that matter for typical use cases. It also requires \textbf{defining} the variables in the associated \mintinline{C++}{.cc} file.
\begin{minted}{C++}
// Defined in foo.cc
const int kMyNumber = 42;
const char kMyString[] = "Hello";
const absl::string_view kMyStringView = "Hello";
\end{minted}
The \mintinline{C++}{ABSL_CONST_INIT} macro ensures each constant is compile-time initialized, but that is all it does. It \emph{does not} make the variable \mintinline{C++}{const} and it \emph{does not} prevent declarations of variables with non-trivial destructors that violate the style guide rules. See mention of the macro in the \hyperref[sec:static-and-global-variables]{style guide}.

You might be tempted to define the variables in the \mintinline{C++}{.cc} file with \mintinline{C++}{constexpr}, but this is \hyperref[subsubsec:non-portable-mistake]{not a portable approach at the moment}.

NOTE: \mintinline{C++}{absl::string_view} is a good way to declare a string constant. The type has a constexpr constructor and a trivial destructor, so it is safe to declare them as global variables. Because a string view knows its length, using them does not require a runtime call to \mintinline{C++}{strlen()}.

\subsubsection{A constexpr Function}
A \mintinline{C++}{constexpr} function that takes no arguments will always return the same value, so it functions as a constant, and can often be used to initialize other constants at compile time. Because all \mintinline{C++}{constexpr} functions are implicitly \mintinline{C++}{inline}, there are no linkage concerns. The primary disadvantage of this approach is the limitations placed on the code in \mintinline{C++}{constexpr} functions. Secondarily, \mintinline{C++}{constexpr} is a non-trivial aspect of the API contract, which has real consequences .
\begin{minted}{C++}
// in foo.h
constexpr int MyNumber() { return 42; }
\end{minted}

\subsection{An Ordinary Function}
When a \mintinline{C++}{constexpr} function isn’t desirable or possible, an ordinary function may be an option. The function in the following example can’t be \mintinline{C++}{constexpr} because it has a static variable:
\begin{minted}{C++}
inline absl::string_view MyString() {
  static constexpr char kHello[] = "Hello";
  return kHello;
}
\end{minted}
\textbf{NOTE}: make sure you use \mintinline{C++}{static constexpr} specifiers when returning array data, such as \mintinline{C++}{char[]} strings, \mintinline{C++}{absl::string_view}, \mintinline{C++}{absl::Span}, etc, to avoid \hyperref[subsubsec:string-view-mistake]{subtle bugs}.

\subsubsection{A \mintinline{C++}{static} Class Member}
Static members of a class are a good option, assuming you are already working with a class. These always have external linkage.
\begin{minted}{C++}
// Declared in foo.h
class Foo {
 public:
  static constexpr int kMyNumber = 42;
  static constexpr char kMyHello[] = "Hello";
};
\end{minted}
Prior to C++17 it was necessary to also provide definitions for these \mintinline{C++}{static} data members in a \mintinline{C++}{.cc} file, but for data members that are both \mintinline{C++}{static} and \mintinline{C++}{constexpr} these are now unnecessary (and deprecated).
\begin{minted}{C++}
// Defined in foo.cc, prior to C++17.
constexpr int Foo::kMyNumber;
constexpr char Foo::kMyHello[];
\end{minted}
It isn’t worth introducing a class just to act as a scope for a bunch of constants. Use one of the other techniques instead.

\subsubsection{Discouraged Alternatives}
\begin{minted}[bgcolor=code_bg_con]{C++}
#define WHATEVER_VALUE 42
\end{minted}
Using the preprocessor is rarely justified, see \hyperref[sec:preprocessor-macros]{the style guide}.
\begin{minted}{C++}
enum : int { kMyNumber = 42 };
\end{minted}
The enum technique used above can be justified in some circumstances. It produces a constant \mintinline{C++}{kMyNumber} that cannot cause the problems talked about in this tip. But the alternatives already listed will be more familiar to most people, and so are generally preferred. Use an \mintinline{C++}{enum} when it makes sense in its own right (for examples, see \hyperref[ch:tip-of-the-week-86]{Tip \#86} \enquote{Enumerating with Class}).

\subsection{Approaches that Work in Source Files}
All of the approaches described above also work within a single \mintinline{C++}{.cc} file, but may be unnecessarily complex. Because constants declared within a source file are visible only inside that file by default (see \href{http://en.cppreference.com/w/cpp/language/storage_duration}{internal linkage rules}), simpler approaches, such as defining \mintinline{C++}{constexpr} variables, often work:
\begin{minted}{C++}
// within a .cc file!
constexpr int kBufferSize = 42;
constexpr char kBufferName[] = "example";
constexpr absl::string_view kOtherBufferName = "other example";
\end{minted}
The above is fine in a \mintinline{C++}{.cc} file but not a header file (see \hyperref[subsubsec:non-portable-mistake]{caveat}). Read that again and commit it to memory. I’ll explain why soon. Long story short: define variables \mintinline{C++}{constexpr} in \mintinline{C++}{.cc} files or declare them \mintinline{C++}{extern const} in header files.

\subsubsection{Within a Header File, Beware!}
Unless you take care to use idioms explained above, \mintinline{C++}{const} and \mintinline{C++}{constexpr} objects are likely to be \emph{different} objects in each translation unit.

This implies:
\begin{itemize}
    \item \textbf{Bugs}: any code that uses the address of a constant is subject to bugs and even the dreaded \enquote{\href{https://en.cppreference.com/w/cpp/language/ub}{undefined behavior}}.
    \item \textbf{Bloat}: each translation unit including your header gets its own copy of the thing. Not such a big deal for simple things like the primitive numeric types. Not so great for strings and bigger data structures.
\end{itemize}
When at namespace scope (i.e. not in a function or in a class), both \mintinline{C++}{const} and \mintinline{C++}{constexpr} objects implicitly have internal linkage (the same linkage used for unnamed-namespace variables and \mintinline{C++}{static} variables not in a function or in a class). The C++ standard guarantees that every translation unit that uses or references the object gets a different \enquote{copy} or \enquote{instantiation} of the object, each at a \emph{different address}.

Within a class, you must additionally declare these objects as \mintinline{C++}{static}, or they will be unchangeable instance variables, rather than unchangeable class variables shared among every instance of the class.

Likewise, within a function, you must declare these objects as \mintinline{C++}{static}, or they will take up space on the stack and be constructed every time the function is called.

\subsubsection{An Example Bug}\label{subsubsec:an-example-bug}
So, is this a real risk? Consider:
\begin{minted}[bgcolor=code_bg_con]{C++}
// Declared in do_something.h
constexpr char kSpecial[] = "special";

// Does something.  Pass kSpecial and it will do something special.
void DoSomething(const char* value);
\end{minted}
\begin{minted}[bgcolor=code_bg_con]{C++}
// Defined in do_something.cc
void DoSomething(const char* value) {
  // Treat pointer equality to kSpecial as a sentinel.
  if (value == kSpecial) {
    // do something special
  } else {
    // do something boring
  }
}
\end{minted}
Notice that this code compares the address of the first char in \mintinline{C++}{kSpecial} to \mintinline{C++}{value} as a kind of magic value for the function. You sometimes see code do this in an effort to short circuit a full string comparison.

This causes a subtle bug. The \mintinline{C++}{kSpecial} array is \mintinline{C++}{constexpr} which implies that it is \mintinline{C++}{static} (with \enquote{internal} linkage). Even though we think of \mintinline{C++}{kSpecial} as \enquote{a constant} – it really isn’t – it’s a whole family of constants, one per translation unit! Calls to \mintinline{C++}{DoSomething(kSpecial)} look like they should do the same thing, but the function takes different code paths depending on where the call occurs.

Any code using a constant array defined in a header file, or code that takes the address of a constant defined in a header file, suffices for this kind of bug. This class of bug is usually seen with string constants, because they are the most common reason to define arrays in header files.

\subsubsection{An Example of Undefined Behavior}
Just tweak the above example, and move \mintinline{C++}{DoSomething} into the header file as an \mintinline{C++}{inline} function. Bingo: now we’ve got undefined behavior, or UB. The language requires all \mintinline{C++}{inline} functions to be defined exactly the same way in every translation unit (source file) – this is part of the language’s \enquote{One Definition Rule}. This particular \mintinline{C++}{DoSomething} implementation references a static variable, so every translation unit actually defines \mintinline{C++}{DoSomething} \emph{differently}, hence the undefined behavior.

Unrelated changes to program code and compilers can change inlining decisions, which can cause undefined behavior like this to change from benign behavior to bug.

\subsubsection{Does this Cause Problems in Practice?}
Yes. In one real-life bug we’ve encountered, the compiler was able to determine that in a particular translation unit (source file), a large static const array defined in a header file was only partially used. Rather than emit the entire array, it optimized away the parts it knew weren’t used. One of the ways the array was partially used was through an inline function declared in a header.

The trouble is, the array was used by other translation units in such a way that the static const array was fully used. For those translation units, the compiler generated a version of the inline function that used the full array.

Then the linker came along. The linker assumed that all instances of the inline function were the same, because the One Definition Rule said they had to be. And it discarded all but one copy of the function - and that was the copy with the partially-optimized array.

This kind of bug is possible when code uses a variable in a way that requires its address to be known. The technical term for this is \enquote{ODR used}. It is difficult to prevent ODR use of variables in modern C++ programs, particularly if those values are passed to template functions (as was the case in the above example).

These bugs do happen and are not easily caught in tests or code review. It pays to stick to safe idioms when defining constants.

\section{Other Common Mistakes}\label{sec:other-common-mistakes}

\subsubsection{Mistake \#1: the Non-Constant Constant}
Seen most often with pointers:
\begin{minted}[bgcolor=code_bg_con]{C++}
const char* kStr = ...;
const Thing* kFoo = ...;
\end{minted}

The \mintinline{C++}{kFoo} above is a pointer to const, but the pointer itself is not a constant. You can assign to it, set it null, etc.
\begin{minted}{C++}
// Corrected.
const Thing* const kFoo = ...;
// This works too.
constexpr const Thing* kFoo = ...;
\end{minted}

\subsubsection{Mistake \#2: the Non-Constant MyString()}\label{subsubsec:string-view-mistake}
Consider this code:
\begin{minted}[bgcolor=code_bg_con]{C++}
inline absl::string_view MyString() {
  return "Hello";  // may return a different value with every call
}
\end{minted}
The address of a string literal constant is allowed to change every time it is evaluated, so the above is subtly wrong because it returns a \mintinline{C++}{string_view} that might have a different \mintinline{C++}{.data()} value for each call. While in many cases this won’t be a problem, it can lead to \hyperref[subsubsec:an-example-bug]{the bug described above}.

Making the \mintinline{C++}{MyString()} \mintinline{C++}{constexpr} does not fix the issue, because the language standard does not say it does. One way to look at this is that a \mintinline{C++}{constexpr} function is just an \mintinline{C++}{inline} function that is allowed to execute at compile time when initializing constant values. At run time it is not different from an \mintinline{C++}{inline} function.
\begin{minted}[bgcolor=code_bg_con]{C++}
constexpr absl::string_view MyString() {
  return "Hello";  // may return a different value with every call
}
\end{minted}

To avoid the bug, use a \mintinline{C++}{static constexpr} variable in a function instead:
\begin{minted}{C++}
inline absl::string_view MyString() {
  static constexpr char kHello[] = "Hello";
  return kHello;
}
\end{minted}
Rule of thumb: if your \enquote{constant} is an array type, store it in a function local static before returning it. This fixes its address.

\subsubsection{Mistake \#3: Non-Portable Code}\label{subsubsec:non-portable-mistake}
Some modern C++ features are not yet supported by some major compilers.
\begin{itemize}
    \item In both Clang and GCC, the \mintinline{C++}{static constexpr char kHello[]} array in the \mintinline{C++}{MyString} function \hyperref[subsubsec:string-view-mistake]{above} can be a \mintinline{C++}{static constexpr absl::string_view}. But this won’t compile in Microsoft Visual Studio. If portability is a concern, avoid \mintinline{C++}{constexpr absl::string_view} until we get the \mintinline{C++}{std::string_view} type from C++17.
    \begin{minted}[bgcolor=code_bg_con]{C++}
inline absl::string_view MyString() {
  // Visual Studio refuses to compile this.
  static constexpr absl::string_view kHello = "Hello";
  return kHello;
}
    \end{minted}
    \item For the \mintinline{C++}{extern const} \hyperref[subsec:an-extern-const-variable]{variables declared in header files} the following approach to defining their values is valid according to the standard C++, and would in fact be preferrable to \mintinline{C++}{ABSL_CONST_INIT}, but it is not yet supported by some compilers.
    \begin{minted}[bgcolor=code_bg_con]{C++}
// Defined in foo.cc -- valid C++, but not supported by MSVC 19.
constexpr absl::string_view kOtherBufferName = "other example";
    \end{minted}
    As a workaround for a \mintinline{C++}{constexpr} variable in a \mintinline{C++}{.cc} file you can provide its value to other files through functions.
\end{itemize}

\subsubsection{Mistake \#4: Improperly Initialized Constants}
The style guide has some detailed rules intended to keep us safe from common problems related to run-time initialization of static and global variables. The root issue arises when the initialization of global variable \mintinline{C++}{X} references another global variable \mintinline{C++}{Y}. How can we be sure \mintinline{C++}{Y} itself doesn’t somehow depend on the value of \mintinline{C++}{X}? Cyclic initialization dependencies can easily happen with global variables, especially with those we think of as constants.

This is a pretty thorny area of the language in its own right. \hyperref[sec:static-and-global-variables]{The style guide} is an authoritative reference.

Consider the above links required reading. With a focus on initialization of constants, the phases of initialization can be explained as:
\begin{enumerate}
    \item \textbf{Zero initialization}. This is what initializes otherwise uninitialized static variables to the \enquote{zero} value for the type (e.g. \mintinline{C++}{0}, \mintinline{C++}{0.0}, \mintinline{C++}{'\0'}, \mintinline{C++}{null}, etc.).
    \begin{minted}{C++}
const int kZero;  // this will be zero-initialized to 0
const int kLotsOfZeroes[5000];  // so will all of these
    \end{minted}
    Note that relying on zero initialization is fairly popular in C code but is fairly rare and niche in C++. It is generally clearer to assign variables explicit values, even if the value is zero, which brings us to…
    \item \textbf{Constant initialization}.
    \begin{minted}{C++}
const int kZero = 0;  // this will be constant-initialized to 0
const int kOne = 1;   // this will be constant-initialized to 1
    \end{minted}
    Both \enquote{constant initialization} and \enquote{zero initialization} are called \enquote{static initialization} in the C++ language standard. Both are always safe.
    \item \textbf{Dynamic initialization}.
    \begin{minted}[bgcolor=code_bg_con]{C++}
// This will be dynamically initialized at run-time to
// whatever ArbitraryFunction returns.
const int kArbitrary = ArbitraryFunction();
    \end{minted}
    Dynamic initialization is where most problems happen. The style guide explains why at \hyperref[sec:static-and-global-variables]{Static and Global Variables}.
\end{enumerate}
Note that documents like the Google C++ style guide have historically included dynamic initialization in the broad category of \enquote{static initialization}. The word \enquote{static} applies to a few different concepts in C++, which can be confusing. \enquote{Static initialization} can mean \enquote{initialization of static variables}, which can include run-time computation (dynamic initialization). The language standard uses the term \enquote{static initialization} in a different, narrower, sense: initialization that is done statically or at compile-time.


\section{Initialization Cheat Sheet}\label{sec:initialization-cheat-sheet}
Here is a super-quick constant initialization cheat sheet (not in header files):
\begin{itemize}
    \item \mintinline{C++}{constexpr} guarantees safe constant initialization as well as safe (trivial) destruction. Any \mintinline{C++}{constexpr} variable is entirely fine when defined in a \mintinline{C++}{.cc} file, but is problematic in header files for reasons explained earlier.
    \item \mintinline{C++}{ABSL_CONST_INIT} guarantees safe constant initialization. Unlike \mintinline{C++}{constexpr}, it does not actually make the variable \mintinline{C++}{const}, nor does it ensure the destructor is trivial, so care must still be taken when declaring static variables with it. See again \hyperref[sec:static-and-global-variables]{Static and Global Variables}.
    \item Otherwise, you’re most likely best off using a static variable within a function and returning it. See the \enquote{ordinary function} example shown earlier.
\end{itemize}

\section{Further Reading and Collected Links}\label{sec:further-reading-and-collected-links}
\begin{itemize}
    \item \hyperref[sec:static-and-global-variables]{Static and Global Variables}
    \item \href{http://en.cppreference.com/w/cpp/language/constexpr}{constexpr}
    \item \href{http://en.cppreference.com/w/cpp/language/inline}{inline}
    \item \href{http://en.cppreference.com/w/cpp/language/storage_duration (linkage rules)}{storage duration (linkage rules)}
    \item \href{http://en.cppreference.com/w/cpp/language/}{ub (Undefined Behavior)}
\end{itemize}

\section{Conclusion}\label{sec:conclusion}
The \mintinline{C++}{inline} variable from C++17 can’t come soon enough. Until then all we can do is use the safe idioms that steer us clear of the rough edges.

\begin{itemize}
    \item We conclude that string literals are not required to evaluate to the same object from the following language in \mintinline{C++}{[lex.string]} in the C++17 language standard. Equivalent language is also present in C++11 and C++14.
    \item There is no language in \mintinline{C++}{[lex.string]} describing different behavior in a \mintinline{C++}{constexpr} context.
\end{itemize}
