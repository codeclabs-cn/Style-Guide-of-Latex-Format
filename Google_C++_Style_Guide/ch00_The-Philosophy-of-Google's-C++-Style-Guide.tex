%! Author = codeclabs-cn
%! Date = 2023/1/5

\chapter{The Philosophy of Google's C++ Style Guide\texorpdfstring{\footnote{This is just the main part of Titus Winters's great presentation at \href{https://github.com/CppCon/CppCon2014/tree/master/Presentations/The Philosophy of Google's C++ Style Guide}{CppCon2014}. You can watch the full talk on Youbute \href{https://www.youtube.com/watch?v=NOCElcMcFik}{: The Philosophy of Google's C++ Style Guide}}}{}}
\chapterauthor{Titus Winters (titus@google.com)}

\begin{introduction}
    \item \hyperref[sec:about-us]{About Us}
    \item \hyperref[sec:the-underpinnings-of-googles-c++-style-guide]{The underpinnings of Google’s C++ Style Guide}
    \item \hyperref[sec:the-contentious-rules]{The Contentious Rules}
    \item \hyperref[sec:recap]{Recap}
\end{introduction}

\section{About Us}\label{sec:about-us}
\begin{itemize}
    \item 4K-ish C++ engineers
    \item Shared codebase
    \item Strong testing culture
    \item Good indexer (Kythe)
    \item Wild variance in C++ background
    \item Good code review policies
    \item We expect we’ll be around for a while, and should plan accordingly
\end{itemize}
Most projects check into the same codebase. Most engineers have read access to most code. Most projects use the same infrastructure (libraries, build system, etc).
Code is going to live a long time, and be read many times. We choose explicitly to optimize for the reader, not the writer.

\section{The underpinnings of Google’s C++ Style Guide}\label{sec:the-underpinnings-of-googles-c++-style-guide}

\paragraph*{\#1 Optimize for the Reader, not the Writer}
\begin{itemize}
    \item We’re much more concerned with the experience of code readers.
\end{itemize}

\paragraph*{\#2 Rules Should Pull Their Weight}
\begin{itemize}
    \item We aren’t going to list every single thing you shouldn’t do. Rules for dumb stuff should be handled at a higher level (\enquote{Don’t be clever}).
\end{itemize}

\paragraph*{\#3 Value the Standard, but don’t Idolize}
\begin{itemize}
    \item Tracking the standard is valuable (cppreference.com, stackoverflow,etc). Not everything in the standard is equally good.
\end{itemize}

\paragraph*{\#4 Be Consistent}\mbox{}\newline
Pros:
\begin{itemize}
    \item Consistency allows easier expert chunking.
    \item Consistency allows tooling.
    \item Consistency allows us to stop arguing about stuff that doesn’t matter.
\end{itemize}
Practices:
\begin{itemize}
    \item Include guard naming / formatting
    \item Parameter ordering (input, then output, unless consistency with other things matters)
    \item Namespaces (naming)
    \item Declaration order
    \item 0 and NULL vs. nullptr
    \item Naming
    \item Formatting
    \item \color{red}\sout{Don’t use streams}
\end{itemize}

\paragraph*{\#5 If something unusual is happening, leave explicit evidence for the reader}
\begin{itemize}
    \item Old Example: \enquote{No non-const references} leads to \enquote{The extra \enquote*{\&} means it could be mutated}.
\begin{minted}[escapeinside=||]{C++}
int main(int argc, char** argv) {
  ParseCommandLineFlags(|\colorbox{red}{&}|argc, |\colorbox{red}{&}|argv, true);
}
\end{minted}
    \item New Example: The design of \mintinline[breakanywhere,bgcolor=code_bg_pro]{C++}{std::unique_ptr} makes it fit perfectly into a codebase with pre-C++-style pointers.
    \begin{minted}[escapeinside=||]{C++}
// Taking ownership: new from old.
std::unique_ptr<Foo> my_foo(NewFoo());

// or old from new
Foo* my_foo = NewFoo().release();

// or new from new
std::unique_ptr<Foo> my_foo = NewFoo();

// Yielding ownership (new to old)
TakeFoo(my_foo.release());

// or new to new
TakeFoo(std::move(my_foo));

// or old to new
TakeFoo(std::make_unique<Foo>(my_foo));
    \end{minted}
    \item Rules that help leave a trace for the reader include:
    \begin{itemize}
        \item \mintinline[breakanywhere,bgcolor=code_bg_pro]{C++}{override} or \mintinline[breakanywhere,bgcolor=code_bg_pro]{C++}{final}
        \item Interface classes - Name them with the \enquote{Interface} suffix
        \item Function overloading - If it matters which overload is being called, make it obvious by inspection
        \item No Exceptions - Error handling is explicit
    \end{itemize}
\end{itemize}

\paragraph*{\#6 Avoid constructs that are dangerous or surprising}
\begin{itemize}
    \item Waivers here are probably rare, and would require a strong argument, and probably some comments to mitigate the chance of copy and paste re-using those patterns unsafely. Examples include:
    \begin{itemize}
        \item Static and global variables of complex type (danger at shutdown)
        \item Use override or final (avoid surprise)
        \item Exceptions (dangerous)
    \end{itemize}
    \item Most code should avoid the tricky stuff. Waivers may be granted if justified.
    \begin{itemize}
        \item Avoid macros (non-obvious, complicated)
        \item Template metaprogramming (complicated, often non-obvious)
        \item Non-public inheritance (surprising)
        \item Multiple implementation inheritance (hard to maintain)
    \end{itemize}
\end{itemize}

\paragraph*{\#7 Avoid polluting the global namespace}\mbox{}\newline
Waivers here are unlikely except in very extreme cases.
\begin{itemize}
    \item Put your stuff in a namespace
    \item Don’t \enquote{using} into the global namespace from a header
    \item Inside a .cc: We don’t care much
    \begin{itemize}
        \item Still a distinction between using vs. using namespace
    \end{itemize}
\end{itemize}

\paragraph*{\#8 Concede to optimization and practicalities when necessary}\mbox{}\newline
Sometimes we make rulings just to state that an optimization may be healthy and necessary. (These are usually explicit \enquote{is allowed}.)
\begin{itemize}
    \item Allow forward declarations (\enquote{optimizing} build times)*
    \item Inline functions
    \item Prefer pre-increment (++i)
\end{itemize}

\section{The Contentious Rules}\label{sec:the-contentious-rules}
\paragraph*{There are two (very) contentious rules:}
\begin{itemize}
    \item No non-const references as function arguments
    \item No use of exceptions
\end{itemize}

\subsection{non-const references}
\paragraph*{Three rules apply:}
\begin{itemize}
    \item Consistency
    \item Leave a trace/explicitness
    \item Dangerous/surprising constructs: reference lifetime issues
\end{itemize}

\subsection{no exceptions}
\paragraph*{Some rules apply:}
\begin{itemize}
    \item Value the standard, but don’t idolize
    \item Consistency
    \begin{itemize}
        \item This stems from old compiler bugs, but once that happened ...
    \end{itemize}
    \item Leave a trace
    \item Dangerous/surprising constructs
    \item Avoid hard to maintain constructs
    \begin{itemize}
        \item Consider cases where exception types are changed
    \end{itemize}
    \item Concede to optimization
    \begin{itemize}
        \item On average, code locality matters.
    \end{itemize}
\end{itemize}

\section{Recap}\label{sec:recap}
\begin{itemize}
    \item Have a style guide. Tailor it to your situation.
    \item Use your guide to encourage \enquote{good} and discourage \enquote{bad}.
    \item Re-evaluate.
\end{itemize}







