%! Author = codeclabs-cn
%! Date = 2022/12/28

\chapter{Background}\label{ch:background}

\begin{introduction}
    \item \hyperref[sec:goals-of-the-style-guide]{Goals of the Style Guide}
\end{introduction}

C++ is one of the main development languages used by many of Google\rq s open-source projects. As every C++ programmer knows, the language has many powerful features, but this power brings with it complexity, which in turn can make code more bug-prone and harder to read and maintain.

The goal of this guide is to manage this complexity by describing in detail the dos and don\rq ts of writing C++ code . These rules exist to keep the code base manageable while still allowing coders to use C++ language features productively.

\emph{Style}, also known as readability, is what we call the conventions that govern our C++ code. The term Style is a bit of a misnomer, since these conventions cover far more than just source file formatting.

Most open-source projects developed by Google conform to the requirements in this guide.

Note that this guide is not a C++ tutorial: we assume that the reader is familiar with the language.

\subimport{./}{Google_C++_Style_Guide/ch01/section_01.tex}