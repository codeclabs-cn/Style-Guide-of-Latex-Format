%! Author = codeclabs-cn
%! Date = 2022/12/28

\chapter{Header Files}\label{ch:header-files}

\begin{introduction}
    \item \hyperref[sec:self-contained-headers]{Self-contained Headers}
    \item \hyperref[sec:the-define-guard]{The} \mintinline[breakanywhere,bgcolor=code_bg_pro]{C++}{#define} \hyperref[sec:the-define-guard]{Guard}
    \item \hyperref[sec:include-what-you-use]{Include What You Use}
    \item \hyperref[sec:forward_declarations]{Forward Declarations}
    \item \hyperref[sec:inline-functions]{Inline Functions}
    \item \hyperref[sec:names-and-order-of-includes]{Names and Order of Includes}
\end{introduction}

In general, every \mintinline[breakanywhere,bgcolor=code_bg_pro]{C++}{.cc} file should have an associated \mintinline[breakanywhere,bgcolor=code_bg_pro]{C++}{.h} file. There are some common exceptions, such as unit tests and small \mintinline[breakanywhere,bgcolor=code_bg_pro]{C++}{.cc} files containing just a \mintinline[breakanywhere,bgcolor=code_bg_pro]{C++}{main()} function.

Correct use of header files can make a huge difference to the readability, size and performance of your code.

The following rules will guide you through the various pitfalls of using header files.

\subimport{./}{Google_C++_Style_Guide/ch03/section_01.tex}
\subimport{./}{Google_C++_Style_Guide/ch03/section_02.tex}
\subimport{./}{Google_C++_Style_Guide/ch03/section_03.tex}
\subimport{./}{Google_C++_Style_Guide/ch03/section_04.tex}
\subimport{./}{Google_C++_Style_Guide/ch03/section_05.tex}
\subimport{./}{Google_C++_Style_Guide/ch03/section_06.tex}










