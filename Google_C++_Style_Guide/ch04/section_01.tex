%! Author = codeclabs-cn
%! Date = 2022/12/29

\section{Namespaces}\label{sec:namespaces}
With few exceptions, place code in a namespace. Namespaces should have unique names based on the project name, and possibly its path. Do not use \emph{using-directives} (e.g., \mintinline[breakanywhere,bgcolor=code_bg_pro]{C++}{using namespace foo}). Do not use inline namespaces. For unnamed namespaces, see \hyperref[sec:internal-linkage]{Internal Linkage}.

\subsection{Definition}
Namespaces subdivide the global scope into distinct, named scopes, and so are useful for preventing name collisions in the global scope.

\subsection{Pros}
Namespaces provide a method for preventing name conflicts in large programs while allowing most code to use reasonably short names.

For example, if two different projects have a class \mintinline[breakanywhere,bgcolor=code_bg_pro]{C++}{Foo} in the global scope, these symbols may collide at compile time or at runtime. If each project places their code in a namespace, \mintinline[breakanywhere,bgcolor=code_bg_pro]{C++}{project1::Foo} and \mintinline[breakanywhere,bgcolor=code_bg_pro]{C++}{project2::Foo} are now distinct symbols that do not collide, and code within each project's namespace can continue to refer to \mintinline[breakanywhere,bgcolor=code_bg_pro]{C++}{Foo} without the prefix.

Inline namespaces automatically place their names in the enclosing scope. Consider the following snippet, for example:
% \vspace{-\baselineskip}
\begin{minted}{C++}
namespace outer {
inline namespace inner {
  void foo();
}  // namespace inner
}  // namespace outer
\end{minted}

The expressions \mintinline[breakanywhere,bgcolor=code_bg_pro]{C++}{outer::inner::foo()} and \mintinline[breakanywhere,bgcolor=code_bg_pro]{C++}{outer::foo()} are interchangeable. Inline namespaces are primarily intended for ABI compatibility across versions.

\subsection{Cons}
Namespaces can be confusing, because they complicate the mechanics of figuring out what definition a name refers to.

Inline namespaces, in particular, can be confusing because names aren't actually restricted to the namespace where they are declared. They are only useful as part of some larger versioning policy.

In some contexts, it's necessary to repeatedly refer to symbols by their fully-qualified names. For deeply-nested namespaces, this can add a lot of clutter.

\subsection{Decision}
Namespaces should be used as follows:
\begin{itemize}
    \item Follow the rules on \hyperref[subsec:namespace-names]{Namespace Names}.
    \item Terminate multi-line namespaces with comments as shown in the given examples.
    \item Namespaces wrap the entire source file after includes, \href{https://gflags.github.io/gflags/}{gflags} definitions/declarations and forward declarations of classes from other namespaces.
    % \vspace{-\baselineskip}
    \begin{minted}[mathescape,
        linenos,
        numbersep=5pt,
        autogobble, % 左对齐
        breaklines,
        frame=lines,
        framesep=2mm]{C++}
// In the .h file
namespace mynamespace {

// All declarations are within the namespace scope.
// Notice the lack of indentation.
class MyClass {
 public:
  ...
  void Foo();
};

}  // namespace mynamespace
    \end{minted}
    % \vspace{-\baselineskip}
    \begin{minted}[mathescape,
        linenos,
        numbersep=5pt,
        autogobble, % 左对齐
        breaklines,
        frame=lines,
        framesep=2mm]{C++}
// In the .cc file
namespace mynamespace {

// Definition of functions is within scope of the namespace.
void MyClass::Foo() {
  ...
}

}  // namespace mynamespace
    \end{minted}
    More complex \mintinline[breakanywhere,bgcolor=code_bg_pro]{C++}{.cc} files might have additional details, like flags or using-declarations.
    % \vspace{-\baselineskip}
    \begin{minted}[mathescape,
        linenos,
        numbersep=5pt,
        autogobble, % 左对齐
        breaklines,
        frame=lines,
        framesep=2mm]{C++}
#include "a.h"

ABSL_FLAG(bool, someflag, false, "a flag");

namespace mynamespace {

using ::foo::Bar;

...code for mynamespace...    // Code goes against the left margin.

}  // namespace mynamespace
    \end{minted}
    \item To place generated protocol message code in a namespace, use the \mintinline[breakanywhere,bgcolor=code_bg_pro]{C++}{package} specifier in the \mintinline[breakanywhere,bgcolor=code_bg_pro]{C++}{.proto} file. See \href{https://developers.google.com/protocol-buffers/docs/reference/cpp-generated#package}{Protocol Buffer Packages} for details.
    \item Do not declare anything in namespace \mintinline[breakanywhere,bgcolor=code_bg_pro]{C++}{std}, including forward declarations of standard library classes. Declaring entities in namespace \mintinline[breakanywhere,bgcolor=code_bg_pro]{C++}{std} is undefined behavior, i.e., not portable. To declare entities from the standard library, include the appropriate header file.
    \item You may not use a \emph{using-directive} to make all names from a namespace available.
    % \vspace{-\baselineskip}
    \begin{minted}[mathescape,
        bgcolor=code_bg_con,
        linenos,
        numbersep=5pt,
        autogobble, % 左对齐
        breaklines,
        frame=lines,
        framesep=2mm]{C++}
// Forbidden -- This pollutes the namespace.
using namespace foo;
    \end{minted}
    \item Do not use \emph{Namespace aliases} at namespace scope in header files except in explicitly marked internal-only namespaces, because anything imported into a namespace in a header file becomes part of the public API exported by that file.
    % \vspace{-\baselineskip}
    \begin{minted}[mathescape,
        bgcolor=code_bg_pro,
        linenos,
        numbersep=5pt,
        autogobble, % 左对齐
        breaklines,
        frame=lines,
        framesep=2mm]{C++}
// Shorten access to some commonly used names in .cc files.
namespace baz = ::foo::bar::baz;
    \end{minted}
    % \vspace{-\baselineskip}
    \begin{minted}[mathescape,
        bgcolor=code_bg_pro,
        linenos,
        numbersep=5pt,
        autogobble, % 左对齐
        breaklines,
        frame=lines,
        framesep=2mm]{C++}
// Shorten access to some commonly used names (in a .h file).
namespace librarian {
namespace impl {  // Internal, not part of the API.
namespace sidetable = ::pipeline_diagnostics::sidetable;
}  // namespace impl

inline void my_inline_function() {
  // namespace alias local to a function (or method).
  namespace baz = ::foo::bar::baz;
  ...
}
}  // namespace librarian
    \end{minted}
    \item Do not use inline namespaces.
\end{itemize}