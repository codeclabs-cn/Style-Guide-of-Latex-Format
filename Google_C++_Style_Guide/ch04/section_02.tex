%! Author = codeclabs-cn
%! Date = 2022/12/29


\section{Internal Linkage}\label{sec:internal-linkage}
When definitions in a \mintinline[breakanywhere,bgcolor=code_bg_pro]{C++}{.cc} file do not need to be referenced outside that file, give them internal linkage by placing them in an unnamed namespace or declaring them \mintinline[breakanywhere,bgcolor=code_bg_pro]{C++}{static}. Do not use either of these constructs in \mintinline[breakanywhere,bgcolor=code_bg_pro]{C++}{.h} files.

\subsection{Definition}
All declarations can be given internal linkage by placing them in unnamed namespaces. Functions and variables can also be given internal linkage by declaring them \mintinline[breakanywhere,bgcolor=code_bg_pro]{C++}{static}. This means that anything you're declaring can't be accessed from another file. If a different file declares something with the same name, then the two entities are completely independent.

\subsection{Decision}
Use of internal linkage in \mintinline[breakanywhere,bgcolor=code_bg_pro]{C++}{.cc} files is encouraged for all code that does not need to be referenced elsewhere. Do not use internal linkage in \mintinline[breakanywhere,bgcolor=code_bg_pro]{C++}{.h} files.

Format unnamed namespaces like named namespaces. In the terminating comment, leave the namespace name empty:
% \vspace{-\baselineskip}
\begin{minted}{C++}
namespace {
...
}  // namespace
\end{minted}
