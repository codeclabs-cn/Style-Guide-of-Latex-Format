%! Author = codeclabs-cn
%! Date = 2022/12/29

\section{Copyable and Movable Types}\label{sec:copyable-and-movable-types}
A class's public API must make clear whether the class is copyable, move-only, or neither copyable nor movable. Support copying and/or moving if these operations are clear and meaningful for your type.

\subsection{Definition}
A movable type is one that can be initialized and assigned from temporaries.

A copyable type is one that can be initialized or assigned from any other object of the same type (so is also movable by definition), with the stipulation that the value of the source does not change. \mintinline[breakanywhere,bgcolor=code_bg_pro]{C++}{std::unique_ptr<int>} is an example of a movable but not copyable type (since the value of the source \mintinline[breakanywhere,bgcolor=code_bg_pro]{C++}{std::unique_ptr<int>} must be modified during assignment to the destination). \mintinline[breakanywhere,bgcolor=code_bg_pro]{C++}{int} and \mintinline[breakanywhere,bgcolor=code_bg_pro]{C++}{std::string} are examples of movable types that are also copyable. (For \mintinline[breakanywhere,bgcolor=code_bg_pro]{C++}{int}, the move and copy operations are the same; for \mintinline[breakanywhere,bgcolor=code_bg_pro]{C++}{std::string}, there exists a move operation that is less expensive than a copy.)

For user-defined types, the copy behavior is defined by the copy constructor and the copy-assignment operator. Move behavior is defined by the move constructor and the move-assignment operator, if they exist, or by the copy constructor and the copy-assignment operator otherwise.

The copy/move constructors can be implicitly invoked by the compiler in some situations, e.g., when passing objects by value.

\subsection{Pros}
Objects of copyable and movable types can be passed and returned by value, which makes APIs simpler, safer, and more general. Unlike when passing objects by pointer or reference, there's no risk of confusion over ownership, lifetime, mutability, and similar issues, and no need to specify them in the contract. It also prevents non-local interactions between the client and the implementation, which makes them easier to understand, maintain, and optimize by the compiler. Further, such objects can be used with generic APIs that require pass-by-value, such as most containers, and they allow for additional flexibility in e.g., type composition.

Copy/move constructors and assignment operators are usually easier to define correctly than alternatives like \mintinline[breakanywhere,bgcolor=code_bg_pro]{C++}{Clone()}, \mintinline[breakanywhere,bgcolor=code_bg_pro]{C++}{CopyFrom()} or \mintinline[breakanywhere,bgcolor=code_bg_pro]{C++}{Swap()}, because they can be generated by the compiler, either implicitly or with \mintinline[breakanywhere,bgcolor=code_bg_pro]{C++}{= default}. They are concise, and ensure that all data members are copied. Copy and move constructors are also generally more efficient, because they don't require heap allocation or separate initialization and assignment steps, and they're eligible for optimizations such as \href{http://en.cppreference.com/w/cpp/language/copy_elision}{copy elision}.

Move operations allow the implicit and efficient transfer of resources out of rvalue objects. This allows a plainer coding style in some cases.


\subsection{Cons}
Some types do not need to be copyable, and providing copy operations for such types can be confusing, nonsensical, or outright incorrect. Types representing singleton objects (\mintinline[breakanywhere,bgcolor=code_bg_pro]{C++}{Registerer}), objects tied to a specific scope (\mintinline[breakanywhere,bgcolor=code_bg_pro]{C++}{Cleanup}), or closely coupled to object identity (\mintinline[breakanywhere,bgcolor=code_bg_pro]{C++}{Mutex}) cannot be copied meaningfully. Copy operations for base class types that are to be used polymorphically are hazardous, because use of them can lead to \href{https://en.wikipedia.org/wiki/Object_slicing}{object slicing}. Defaulted or carelessly-implemented copy operations can be incorrect, and the resulting bugs can be confusing and difficult to diagnose.

Copy constructors are invoked implicitly, which makes the invocation easy to miss. This may cause confusion for programmers used to languages where pass-by-reference is conventional or mandatory. It may also encourage excessive copying, which can cause performance problems.

\subsection{Decision}
Every class's public interface must make clear which copy and move operations the class supports. This should usually take the form of explicitly declaring and/or deleting the appropriate operations in the \mintinline[breakanywhere,bgcolor=code_bg_pro]{C++}{public} section of the declaration.

Specifically, a copyable class should explicitly declare the copy operations, a move-only class should explicitly declare the move operations, and a non-copyable/movable class should explicitly delete the copy operations. A copyable class may also declare move operations in order to support efficient moves. Explicitly declaring or deleting all four copy/move operations is permitted, but not required. If you provide a copy or move assignment operator, you must also provide the corresponding constructor.
% \vspace{-\baselineskip}
\begin{minted}{C++}
class Copyable {
 public:
  Copyable(const Copyable& other) = default;
  Copyable& operator=(const Copyable& other) = default;

  // The implicit move operations are suppressed by the declarations above.
  // You may explicitly declare move operations to support efficient moves.
};

class MoveOnly {
 public:
  MoveOnly(MoveOnly&& other) = default;
  MoveOnly& operator=(MoveOnly&& other) = default;

  // The copy operations are implicitly deleted, but you can
  // spell that out explicitly if you want:
  MoveOnly(const MoveOnly&) = delete;
  MoveOnly& operator=(const MoveOnly&) = delete;
};

class NotCopyableOrMovable {
 public:
  // Not copyable or movable
  NotCopyableOrMovable(const NotCopyableOrMovable&) = delete;
  NotCopyableOrMovable& operator=(const NotCopyableOrMovable&)
      = delete;

  // The move operations are implicitly disabled, but you can
  // spell that out explicitly if you want:
  NotCopyableOrMovable(NotCopyableOrMovable&&) = delete;
  NotCopyableOrMovable& operator=(NotCopyableOrMovable&&)
      = delete;
};
\end{minted}

These declarations/deletions can be omitted only if they are obvious:

\begin{itemize}
    \item If the class has no \mintinline[breakanywhere,bgcolor=code_bg_pro]{C++}{private} section, like a \hyperref[sec:structs-vs.-classes]{struct} or an interface-only base class, then the copyability/movability can be determined by the copyability/movability of any public data members.
    \item If a base class clearly isn't copyable or movable, derived classes naturally won't be either. An interface-only base class that leaves these operations implicit is not sufficient to make concrete subclasses clear.
    \item Note that if you explicitly declare or delete either the constructor or assignment operation for copy, the other copy operation is not obvious and must be declared or deleted. Likewise for move operations.
\end{itemize}

A type should not be copyable/movable if the meaning of copying/moving is unclear to a casual user, or if it incurs unexpected costs. Move operations for copyable types are strictly a performance optimization and are a potential source of bugs and complexity, so avoid defining them unless they are significantly more efficient than the corresponding copy operations. If your type provides copy operations, it is recommended that you design your class so that the default implementation of those operations is correct. Remember to review the correctness of any defaulted operations as you would any other code.

To eliminate the risk of slicing, prefer to make base classes abstract, by making their constructors protected, by declaring their destructors protected, or by giving them one or more pure virtual member functions. Prefer to avoid deriving from concrete classes.
