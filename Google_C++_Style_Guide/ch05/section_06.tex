%! Author = codeclabs-cn
%! Date = 2022/12/29

\section{Inheritance}\label{sec:inheritance}
Composition is often more appropriate than inheritance. When using inheritance, make it \mintinline[breakanywhere,bgcolor=code_bg_pro]{C++}{public}.

\subsection{Definition}
When a sub-class inherits from a base class, it includes the definitions of all the data and operations that the base class defines. \enquote{Interface inheritance} is inheritance from a pure abstract base class (one with no state or defined methods); all other inheritance is \enquote{implementation inheritance}.

\subsection{Pros}
Implementation inheritance reduces code size by re-using the base class code as it specializes an existing type. Because inheritance is a compile-time declaration, you and the compiler can understand the operation and detect errors. Interface inheritance can be used to programmatically enforce that a class expose a particular API. Again, the compiler can detect errors, in this case, when a class does not define a necessary method of the API.

\subsection{Cons}
For implementation inheritance, because the code implementing a sub-class is spread between the base and the sub-class, it can be more difficult to understand an implementation. The sub-class cannot override functions that are not virtual, so the sub-class cannot change implementation.

Multiple inheritance is especially problematic, because it often imposes a higher performance overhead (in fact, the performance drop from single inheritance to multiple inheritance can often be greater than the performance drop from ordinary to virtual dispatch), and because it risks leading to \enquote{diamond} inheritance patterns, which are prone to ambiguity, confusion, and outright bugs.

\subsection{Decision}
\begin{itemize}
    \item All inheritance should be \mintinline[breakanywhere,bgcolor=code_bg_pro]{C++}{public}. If you want to do private inheritance, you should be including an instance of the base class as a member instead. You may use \mintinline[breakanywhere,bgcolor=code_bg_pro]{C++}{final} on classes when you don't intend to support using them as base classes.
    \item Do not overuse implementation inheritance. Composition is often more appropriate. Try to restrict use of inheritance to the \enquote{is-a} case: \mintinline[breakanywhere,bgcolor=code_bg_pro]{C++}{Bar} subclasses \mintinline[breakanywhere,bgcolor=code_bg_pro]{C++}{Foo} if it can reasonably be said that \mintinline[breakanywhere,bgcolor=code_bg_pro]{C++}{Bar} \enquote{s a kind of} \mintinline[breakanywhere,bgcolor=code_bg_pro]{C++}{Foo}.
    \item Limit the use of \mintinline[breakanywhere,bgcolor=code_bg_pro]{C++}{protected} to those member functions that might need to be accessed from subclasses. Note that \hyperref[sec:access-control]{data members should be private}.
    \item Explicitly annotate overrides of virtual functions or virtual destructors with exactly one of an \mintinline[breakanywhere,bgcolor=code_bg_pro]{C++}{override} or (less frequently) \mintinline[breakanywhere,bgcolor=code_bg_pro]{C++}{final} specifier. Do not use \mintinline[breakanywhere,bgcolor=code_bg_pro]{C++}{virtual} when declaring an \mintinline[breakanywhere,bgcolor=code_bg_pro]{C++}{override}. Rationale: A function or destructor marked \mintinline[breakanywhere,bgcolor=code_bg_pro]{C++}{override} or \mintinline[breakanywhere,bgcolor=code_bg_pro]{C++}{final} that is not an override of a base class virtual function will not compile, and this helps catch common errors. The specifiers serve as documentation; if no specifier is present, the reader has to check all ancestors of the class in question to determine if the function or destructor is virtual or not.
    \item Multiple inheritance is permitted, but multiple \emph{implementation} inheritance is strongly discouraged.
\end{itemize}
