%! Author = codeclabs-cn
%! Date = 2022/12/29

\section{Inputs and Outputs}\label{sec:inputs-and-outputs}
The output of a C++ function is naturally provided via a return value and sometimes via output parameters (or in/out parameters).

Prefer using return values over output parameters: they improve readability, and often provide the same or better performance.

Prefer to return by value or, failing that, return by reference. Avoid returning a pointer unless it can be \mintinline[breakanywhere,bgcolor=code_bg_pro]{C++}{null}.

Parameters are either inputs to the function, outputs from the function, or both. Non-optional input parameters should usually be values or \mintinline[breakanywhere,bgcolor=code_bg_pro]{C++}{const} references, while non-optional output and input/output parameters should usually be references (which cannot be \mintinline[breakanywhere,bgcolor=code_bg_pro]{C++}{null}). Generally, use \mintinline[breakanywhere,bgcolor=code_bg_pro]{C++}{std::optional} to represent optional by-value inputs, and use a \mintinline[breakanywhere,bgcolor=code_bg_pro]{C++}{const} pointer when the non-optional form would have used a reference. Use non-\mintinline[breakanywhere,bgcolor=code_bg_pro]{C++}{const} pointers to represent optional outputs and optional input/output parameters.

Avoid defining functions that require a \mintinline[breakanywhere,bgcolor=code_bg_pro]{C++}{const} reference parameter to outlive the call, because \mintinline[breakanywhere,bgcolor=code_bg_pro]{C++}{const} reference parameters bind to temporaries. Instead, find a way to eliminate the lifetime requirement (for example, by copying the parameter), or pass it by \mintinline[breakanywhere,bgcolor=code_bg_pro]{C++}{const} pointer and document the lifetime and non-null requirements.

When ordering function parameters, put all input-only parameters before any output parameters. In particular, do not add new parameters to the end of the function just because they are new; place new input-only parameters before the output parameters. This is not a hard-and-fast rule. Parameters that are both input and output muddy the waters, and, as always, consistency with related functions may require you to bend the rule. Variadic functions may also require unusual parameter ordering.
