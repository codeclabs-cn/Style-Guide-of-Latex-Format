%! Author = codeclabs-cn
%! Date = 2022/12/29

\section{Casting}\label{sec:casting}
Use C++-style casts like \mintinline[breakanywhere,bgcolor=code_bg_pro]{C++}{static_cast<float>(double_value)}, or brace initialization for conversion of arithmetic types like \mintinline[breakanywhere,bgcolor=code_bg_pro]{C++}{int64_t y = int64_t{1} << 42}. Do not use cast formats like \mintinline[breakanywhere,bgcolor=code_bg_pro]{C++}{(int)x} unless the cast is to void. You may use cast formats like \mintinline[breakanywhere,bgcolor=code_bg_pro]{C++}{T(x)} only when \mintinline[breakanywhere,bgcolor=code_bg_pro]{C++}{T} is a class type.

\subsection{Definition}
C++ introduced a different cast system from C that distinguishes the types of cast operations.

\subsection{Pros}
The problem with C casts is the ambiguity of the operation; sometimes you are doing a \emph{conversion} (e.g., \mintinline[breakanywhere,bgcolor=code_bg_pro]{C++}{(int)3.5}) and sometimes you are doing a \emph{cast} (e.g., \mintinline[breakanywhere,bgcolor=code_bg_pro]{C++}{(int)"hello"}). Brace initialization and C++ casts can often help avoid this ambiguity. Additionally, C++ casts are more visible when searching for them.

\subsection{Cons}
The C++-style cast syntax is verbose and cumbersome.

\subsection{Decision}
In general, do not use C-style casts. Instead, use these C++-style casts when explicit type conversion is necessary.
\begin{itemize}
    \item Use brace initialization to convert arithmetic types (e.g., \mintinline[breakanywhere,bgcolor=code_bg_pro]{C++}{int64_t{x}}). This is the safest approach because code will not compile if conversion can result in information loss. The syntax is also concise.
    \item Use \mintinline[breakanywhere,bgcolor=code_bg_pro]{C++}{absl::implicit_cast} to safely cast up a type hierarchy, e.g., casting a \mintinline[breakanywhere,bgcolor=code_bg_pro]{C++}{Foo*} to a \mintinline[breakanywhere,bgcolor=code_bg_pro]{C++}{SuperclassOfFoo*} or casting a \mintinline[breakanywhere,bgcolor=code_bg_pro]{C++}{Foo*} to a \mintinline[breakanywhere,bgcolor=code_bg_pro]{C++}{const Foo*}. C++ usually does this automatically but some situations need an explicit up-cast, such as use of the \mintinline[breakanywhere,bgcolor=code_bg_pro]{C++}{?:} operator.
    \item Use \mintinline[breakanywhere,bgcolor=code_bg_pro]{C++}{static_cast} as the equivalent of a C-style cast that does value conversion, when you need to explicitly up-cast a pointer from a class to its superclass, or when you need to explicitly cast a pointer from a superclass to a subclass. In this last case, you must be sure your object is actually an instance of the subclass.
    \item Use \mintinline[breakanywhere,bgcolor=code_bg_pro]{C++}{const_cast} to remove the \mintinline[breakanywhere,bgcolor=code_bg_pro]{C++}{const} qualifier (see \hyperref[sec:use-of-const]{const}).
    \item Use \mintinline[breakanywhere,bgcolor=code_bg_pro]{C++}{reinterpret_cast} to do unsafe conversions of pointer types to and from integer and other pointer types, including \mintinline[breakanywhere,bgcolor=code_bg_pro]{C++}{void*}. Use this only if you know what you are doing and you understand the aliasing issues. Also, consider the alternative \mintinline[breakanywhere,bgcolor=code_bg_pro]{C++}{absl::bit_cast}.
    \item Use \mintinline[breakanywhere,bgcolor=code_bg_pro]{C++}{absl::bit_cast} to interpret the raw bits of a value using a different type of the same size (a type pun), such as interpreting the bits of a \mintinline[breakanywhere,bgcolor=code_bg_pro]{C++}{double} as \mintinline[breakanywhere,bgcolor=code_bg_pro]{C++}{int64_t}.
\end{itemize}

See the \hyperref[sec:run-time-type-information-(rtti)]{RTTI section} for guidance on the use of \mintinline[breakanywhere,bgcolor=code_bg_pro]{C++}{dynamic_cast}.
