%! Author = codeclabs-cn
%! Date = 2022/12/29

\section{Preprocessor Macros}\label{sec:preprocessor-macros}
Avoid defining macros, especially in headers; prefer inline functions, enums, and \mintinline[breakanywhere,bgcolor=code_bg_pro]{C++}{const} variables. Name macros with a project-specific prefix. Do not use macros to define pieces of a C++ API.

Macros mean that the code you see is not the same as the code the compiler sees. This can introduce unexpected behavior, especially since macros have global scope.

The problems introduced by macros are especially severe when they are used to define pieces of a C++ API, and still more so for public APIs. Every error message from the compiler when developers incorrectly use that interface now must explain how the macros formed the interface. Refactoring and analysis tools have a dramatically harder time updating the interface. As a consequence, we specifically disallow using macros in this way. For example, avoid patterns like:
% \vspace{-\baselineskip}
\begin{minted}[mathescape,
    bgcolor=code_bg_con,
    linenos,
    numbersep=5pt,
    autogobble, % 左对齐
    breaklines,
    frame=lines,
    framesep=2mm]{C++}
class WOMBAT_TYPE(Foo) {
  // ...

 public:
  EXPAND_PUBLIC_WOMBAT_API(Foo)

  EXPAND_WOMBAT_COMPARISONS(Foo, ==, <)
};
\end{minted}
Luckily, macros are not nearly as necessary in C++ as they are in C. Instead of using a macro to inline performance-critical code, use an inline function. Instead of using a macro to store a constant, use a \mintinline[breakanywhere,bgcolor=code_bg_pro]{C++}{const} variable. Instead of using a macro to \enquote{abbreviate} a long variable name, use a reference. Instead of using a macro to conditionally compile code ... well, don't do that at all (except, of course, for the \mintinline[breakanywhere,bgcolor=code_bg_pro]{C++}{#define} guards to prevent double inclusion of header files). It makes testing much more difficult.

Macros can do things these other techniques cannot, and you do see them in the codebase, especially in the lower-level libraries. And some of their special features (like stringifying, concatenation, and so forth) are not available through the language proper. But before using a macro, consider carefully whether there's a non-macro way to achieve the same result. If you need to use a macro to define an interface, contact your project leads to request a waiver of this rule.

The following usage pattern will avoid many problems with macros; if you use macros, follow it whenever possible:

\begin{itemize}
    \item Don't define macros in a \mintinline[breakanywhere,bgcolor=code_bg_pro]{C++}{.h} file.
    \item \mintinline[breakanywhere,bgcolor=code_bg_pro]{C++}{#define} macros right before you use them, and \mintinline[breakanywhere,bgcolor=code_bg_pro]{C++}{#undef} them right after.
    \item Do not just \mintinline[breakanywhere,bgcolor=code_bg_pro]{C++}{#undef} an existing macro before replacing it with your own; instead, pick a name that's likely to be unique.
    \item Try not to use macros that expand to unbalanced C++ constructs, or at least document that behavior well.
    \item Prefer not using \mintinline[breakanywhere,bgcolor=code_bg_pro]{C++}{##} to generate function/class/variable names.
\end{itemize}

Exporting macros from headers (i.e., defining them in a header without \mintinline[breakanywhere,bgcolor=code_bg_pro]{C++}{#undef}ing them before the end of the header) is extremely strongly discouraged. If you do export a macro from a header, it must have a globally unique name. To achieve this, it must be named with a prefix consisting of your project's namespace name (but upper case).

