%! Author = codeclabs-cn
%! Date = 2022/12/29

\section{Template Metaprogramming}\label{sec:template-metaprogramming}
Avoid complicated template programming.

\subsection{Definition}
Template metaprogramming refers to a family of techniques that exploit the fact that the C++ template instantiation mechanism is Turing complete and can be used to perform arbitrary compile-time computation in the type domain.

\subsection{Pros}
Template metaprogramming allows extremely flexible interfaces that are type safe and high performance. Facilities like \href{https://github.com/google/googletest}{GoogleTest}, \mintinline[breakanywhere,bgcolor=code_bg_pro]{C++}{std::tuple}, \mintinline[breakanywhere,bgcolor=code_bg_pro]{C++}{std::function}, and \mintinline[breakanywhere,bgcolor=code_bg_pro]{C++}{Boost.Spirit} would be impossible without it.

\subsection{Cons}
The techniques used in template metaprogramming are often obscure to anyone but language experts. Code that uses templates in complicated ways is often unreadable, and is hard to debug or maintain.

Template metaprogramming often leads to extremely poor compile time error messages: even if an interface is simple, the complicated implementation details become visible when the user does something wrong.

Template metaprogramming interferes with large scale refactoring by making the job of refactoring tools harder. First, the template code is expanded in multiple contexts, and it's hard to verify that the transformation makes sense in all of them. Second, some refactoring tools work with an AST that only represents the structure of the code after template expansion. It can be difficult to automatically work back to the original source construct that needs to be rewritten.


\subsection{Decision}
Template metaprogramming sometimes allows cleaner and easier-to-use interfaces than would be possible without it, but it's also often a temptation to be overly clever. It's best used in a small number of low level components where the extra maintenance burden is spread out over a large number of uses.

Think twice before using template metaprogramming or other complicated template techniques; think about whether the average member of your team will be able to understand your code well enough to maintain it after you switch to another project, or whether a non-C++ programmer or someone casually browsing the code base will be able to understand the error messages or trace the flow of a function they want to call. If you're using recursive template instantiations or type lists or metafunctions or expression templates, or relying on SFINAE or on the \mintinline[breakanywhere,bgcolor=code_bg_pro]{C++}{sizeof} trick for detecting function overload resolution, then there's a good chance you've gone too far.

If you use template metaprogramming, you should expect to put considerable effort into minimizing and isolating the complexity. You should hide metaprogramming as an implementation detail whenever possible, so that user-facing headers are readable, and you should make sure that tricky code is especially well commented. You should carefully document how the code is used, and you should say something about what the \enquote{generated} code looks like. Pay extra attention to the error messages that the compiler emits when users make mistakes. The error messages are part of your user interface, and your code should be tweaked as necessary so that the error messages are understandable and actionable from a user point of view.
