%! Author = codeclabs-cn
%! Date = 2022/12/29

\section{Nonstandard Extensions}\label{sec:nonstandard-extensions}
Nonstandard extensions to C++ may not be used unless otherwise specified.

\subsection{Definition}
Compilers support various extensions that are not part of standard C++. Such extensions include GCC's \mintinline[breakanywhere,bgcolor=code_bg_pro]{C++}{__attribute__}, intrinsic functions such as \mintinline[breakanywhere,bgcolor=code_bg_pro]{C++}{__builtin_prefetch}, inline assembly, \mintinline[breakanywhere,bgcolor=code_bg_pro]{C++}{__COUNTER__}, \mintinline[breakanywhere,bgcolor=code_bg_pro]{C++}{__PRETTY_FUNCTION__}, compound statement expressions (e.g., \mintinline[breakanywhere,bgcolor=code_bg_pro]{C++}{foo = (\{ int x; Bar(\&x); x \})}, variable-length arrays and \mintinline[breakanywhere,bgcolor=code_bg_pro]{C++}{alloca()}, and the \enquote{Elvis Operator} \mintinline[breakanywhere,bgcolor=code_bg_pro]{C++}{a?:b}.

\subsection{Pros}
\begin{itemize}
\item Nonstandard extensions may provide useful features that do not exist in standard C++.
\item Important performance guidance to the compiler can only be specified using extensions.
\end{itemize}

\subsection{Cons}
\begin{itemize}
\item Nonstandard extensions do not work in all compilers. Use of nonstandard extensions reduces portability of code.
\item Even if they are supported in all targeted compilers, the extensions are often not well-specified, and there may be subtle behavior differences between compilers.
\item Nonstandard extensions add to the language features that a reader must know to understand the code.
\end{itemize}

\subsection{Decision}
Do not use nonstandard extensions. You may use portability wrappers that are implemented using nonstandard extensions, so long as those wrappers are provided by a designated project-wide portability header.
