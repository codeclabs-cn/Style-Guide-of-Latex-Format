%! Author = codeclabs-cn
%! Date = 2022/12/31

\section{General Naming Rules}\label{sec:general-naming-rules}
Optimize for readability using names that would be clear even to people on a different team.

Use names that describe the purpose or intent of the object. Do not worry about saving horizontal space as it is far more important to make your code immediately understandable by a new reader. Minimize the use of abbreviations that would likely be unknown to someone outside your project (especially acronyms and initialisms). Do not abbreviate by deleting letters within a word. As a rule of thumb, an abbreviation is probably OK if it's listed in Wikipedia. Generally speaking, descriptiveness should be proportional to the name's scope of visibility. For example, \mintinline[breakanywhere,bgcolor=code_bg_pro]{C++}{n} may be a fine name within a 5-line function, but within the scope of a class, it's likely too vague.
% \vspace{-\baselineskip}
\begin{minted}{C++}
class MyClass {
 public:
  int CountFooErrors(const std::vector<Foo>& foos) {
    int n = 0;  // Clear meaning given limited scope and context
    for (const auto& foo : foos) {
      ...
      ++n;
    }
    return n;
  }
  void DoSomethingImportant() {
    std::string fqdn = ...;  // Well-known abbreviation for Fully Qualified Domain Name
  }
 private:
  const int kMaxAllowedConnections = ...;  // Clear meaning within context
};
\end{minted}
% \vspace{-\baselineskip}
\begin{minted}[bgcolor=code_bg_con]{C++}
class MyClass {
 public:
  int CountFooErrors(const std::vector<Foo>& foos) {
    int total_number_of_foo_errors = 0;  // Overly verbose given limited scope and context
    for (int foo_index = 0; foo_index < foos.size(); ++foo_index) {  // Use idiomatic `i`
      ...
      ++total_number_of_foo_errors;
    }
    return total_number_of_foo_errors;
  }
  void DoSomethingImportant() {
    int cstmr_id = ...;  // Deletes internal letters
  }
 private:
  const int kNum = ...;  // Unclear meaning within broad scope
};
\end{minted}
Note that certain universally-known abbreviations are OK, such as i for an iteration variable and T for a template parameter.

For the purposes of the naming rules below, a \enquote{word} is anything that you would write in English without internal spaces. This includes abbreviations, such as acronyms and initialisms. For names written in mixed case (also sometimes referred to as \enquote{\href{https://en.wikipedia.org/wiki/Camel_case}{camel case}} or \enquote{\href{https://en.wiktionary.org/wiki/Pascal_case}{Pascal case}}), in which the first letter of each word is capitalized, prefer to capitalize abbreviations as single words, e.g., \mintinline[breakanywhere,bgcolor=code_bg_pro]{C++}{StartRpc()} rather than \mintinline[breakanywhere,bgcolor=code_bg_pro]{C++}{StartRPC()}.

Template parameters should follow the naming style for their category: type template parameters should follow the rules for \hyperref[sec:type-names]{type names}, and non-type template parameters should follow the rules for \hyperref[sec:variable-names]{variable names}.
