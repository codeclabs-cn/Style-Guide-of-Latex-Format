%! Author = codeclabs-cn
%! Date = 2022/12/31

\section{Variable Names}\label{sec:variable-names}
The names of variables (including function parameters) and data members are all lowercase, with underscores between words. Data members of classes (but not structs) additionally have trailing underscores. For instance: \mintinline[breakanywhere,bgcolor=code_bg_pro]{C++}{a_local_variable}, \mintinline[breakanywhere,bgcolor=code_bg_pro]{C++}{a_struct_data_member}, \mintinline[breakanywhere,bgcolor=code_bg_pro]{C++}{a_class_data_member_}.


    \subsection{Common Variable names}
    For example:
% \vspace{-\baselineskip}
\begin{minted}[mathescape,
        linenos,
        numbersep=5pt,
        autogobble, % 左对齐
        breaklines,
        frame=lines,
        framesep=2mm]{C++}
std::string table_name;
    \end{minted}
% \vspace{-\baselineskip}
    \begin{minted}[bgcolor=code_bg_con]{C++}
std::string tableName;   // Bad - mixed case.
    \end{minted}
    \subsection{Class Data Members}
    Data members of classes, both static and non-static, are named like ordinary nonmember variables, but with a trailing underscore.
% \vspace{-\baselineskip}
\begin{minted}[mathescape,
        linenos,
        numbersep=5pt,
        autogobble, % 左对齐
        breaklines,
        frame=lines,
        framesep=2mm]{C++}
class TableInfo {
  ...
 private:
  std::string table_name_;  // OK - underscore at end.
  static Pool<TableInfo>* pool_;  // OK.
};
    \end{minted}
    \subsection{Struct Data Members}
    Data members of structs, both static and non-static, are named like ordinary nonmember variables. They do not have the trailing underscores that data members in classes have.
% \vspace{-\baselineskip}
\begin{minted}[mathescape,
        linenos,
        numbersep=5pt,
        autogobble, % 左对齐
        breaklines,
        frame=lines,
        framesep=2mm]{C++}
struct UrlTableProperties {
  std::string name;
  int num_entries;
  static Pool<UrlTableProperties>* pool;
};
    \end{minted}
    See \hyperref[sec:structs-vs.-classes]{Structs vs. Classes} for a discussion of when to use a struct versus a class.
    \subsection{Constant Names}\label{subsec:constant-names}
    Variables declared constexpr or const, and whose value is fixed for the duration of the program, are named with a leading \enquote{k} followed by mixed case. Underscores can be used as separators in the rare cases where capitalization cannot be used for separation. For example:
% \vspace{-\baselineskip}
\begin{minted}[mathescape,
        linenos,
        numbersep=5pt,
        autogobble, % 左对齐
        breaklines,
        frame=lines,
        framesep=2mm]{C++}
const int kDaysInAWeek = 7;
const int kAndroid8_0_0 = 24;  // Android 8.0.0
    \end{minted}
    All such variables with static storage duration (i.e., statics and globals, see \href{http://en.cppreference.com/w/cpp/language/storage_duration#Storage_duration}{Storage Duration} for details) should be named this way. This convention is optional for variables of other storage classes, e.g., automatic variables, otherwise the usual variable naming rules apply.
    \subsection{Function Names}
    Regular functions have mixed case; accessors and mutators may be named like variables.

    Ordinarily, functions should start with a capital letter and have a capital letter for each new word.
% \vspace{-\baselineskip}
\begin{minted}[mathescape,
        linenos,
        numbersep=5pt,
        autogobble, % 左对齐
        breaklines,
        frame=lines,
        framesep=2mm]{C++}
AddTableEntry()
DeleteUrl()
OpenFileOrDie()
    \end{minted}
    (The same naming rule applies to class- and namespace-scope constants that are exposed as part of an API and that are intended to look like functions, because the fact that they're objects rather than functions is an unimportant implementation detail.)

    Accessors and mutators (get and set functions) may be named like variables. These often correspond to actual member variables, but this is not required. For example, \mintinline[breakanywhere,bgcolor=code_bg_pro]{C++}{int count()} and void \mintinline[breakanywhere,bgcolor=code_bg_pro]{C++}{set_count(int count)}.
    \subsection{Namespace Names}\label{subsec:namespace-names}
    Namespace names are all lower-case, with words separated by underscores. Top-level namespace names are based on the project name . Avoid collisions between nested namespaces and well-known top-level namespaces.
    The name of a top-level namespace should usually be the name of the project or team whose code is contained in that namespace. The code in that namespace should usually be in a directory whose basename matches the namespace name (or in subdirectories thereof).

    Keep in mind that the \hyperref[sec:general-naming-rules]{rule against abbreviated names} applies to namespaces just as much as variable names. Code inside the namespace seldom needs to mention the namespace name, so there's usually no particular need for abbreviation anyway.

    Avoid nested namespaces that match well-known top-level namespaces. Collisions between namespace names can lead to surprising build breaks because of name lookup rules. In particular, do not create any nested \mintinline[breakanywhere,bgcolor=code_bg_pro]{C++}{std} namespaces. Prefer unique project identifiers ()\mintinline[breakanywhere,bgcolor=code_bg_pro]{C++}{websearch::index, websearch::index_util}) over collision-prone names like \mintinline[breakanywhere,bgcolor=code_bg_pro]{C++}{websearch::util}. Also avoid overly deep nesting namespaces (\hyperref[ch:tip-of-the-week-130]{TotW \#130}).

    For \mintinline[breakanywhere,bgcolor=code_bg_pro]{C++}{internal} namespaces, be wary of other code being added to the same \mintinline[breakanywhere,bgcolor=code_bg_pro]{C++}{internal} namespace causing a collision (internal helpers within a team tend to be related and may lead to collisions). In such a situation, using the filename to make a unique internal name is helpful (\mintinline[breakanywhere,bgcolor=code_bg_pro]{C++}{websearch::index::frobber_internal} for use in \mintinline[breakanywhere,bgcolor=code_bg_pro]{C++}{frobber.h}).
    \subsection{Enumerator Names}
    Enumerators (for both scoped and unscoped enums) should be named like \hyperref[subsec:constant-names]{constants}, not like \hyperref[subsec:macro-names]{macros}. That is, use \mintinline[breakanywhere,bgcolor=code_bg_pro]{C++}{kEnumName} not \mintinline[breakanywhere,bgcolor=code_bg_pro]{C++}{ENUM_NAME}.
% \vspace{-\baselineskip}
\begin{minted}[mathescape,
        linenos,
        numbersep=5pt,
        autogobble, % 左对齐
        breaklines,
        frame=lines,
        framesep=2mm]{C++}
enum class UrlTableError {
  kOk = 0,
  kOutOfMemory,
  kMalformedInput,
};
    \end{minted}
% \vspace{-\baselineskip}
    \begin{minted}[mathescape,
        bgcolor=code_bg_con,
        linenos,
        numbersep=5pt,
        autogobble, % 左对齐
        breaklines,
        frame=lines,
        framesep=2mm]{C++}
enum class AlternateUrlTableError {
  OK = 0,
  OUT_OF_MEMORY = 1,
  MALFORMED_INPUT = 2,
};
    \end{minted}
    Until January 2009, the style was to name enum values like \hyperref[subsec:macro-names]{macros}. This caused problems with name collisions between enum values and macros. Hence, the change to prefer constant-style naming was put in place. New code should use constant-style naming.
    \subsection{Macro Names}\label{subsec:macro-names}
    You're not really going to \hyperref[sec:preprocessor-macros]{define a macro}, are you? If you do, they're like this: \mintinline[breakanywhere,bgcolor=code_bg_pro]{C++}{MY_MACRO_THAT_SCARES_SMALL_CHILDREN_AND_ADULTS_ALIKE}.

    Please see the description of \hyperref[sec:preprocessor-macros]{macros}; in general macros should \emph{not} be used. However, if they are absolutely needed, then they should be named with all capitals and underscores.
% \vspace{-\baselineskip}
\begin{minted}[mathescape,
        linenos,
        numbersep=5pt,
        autogobble, % 左对齐
        breaklines,
        frame=lines,
        framesep=2mm]{C++}
#define ROUND(x) ...
#define PI_ROUNDED 3.0
    \end{minted}
    \subsection{Exceptions to Naming Rules}
    If you are naming something that is analogous to an existing C or C++ entity then you can follow the existing naming convention scheme.

    \begin{itemize}
        \item \mintinline[breakanywhere,bgcolor=code_bg_pro]{C++}{bigopen()}: function name, follows form of \mintinline[breakanywhere,bgcolor=code_bg_pro]{C++}{open()}
        \item \mintinline[breakanywhere,bgcolor=code_bg_pro]{C++}{uint}: \mintinline[breakanywhere,bgcolor=code_bg_pro]{C++}{typedef}
        \item \mintinline[breakanywhere,bgcolor=code_bg_pro]{C++}{bigpos}: \mintinline[breakanywhere,bgcolor=code_bg_pro]{C++}{struct} or \mintinline[breakanywhere,bgcolor=code_bg_pro]{C++}{class}, follows form of \mintinline[breakanywhere,bgcolor=code_bg_pro]{C++}{pos}
        \item \mintinline[breakanywhere,bgcolor=code_bg_pro]{C++}{sparse_hash_map}: STL-like entity; follows STL naming conventions
        \item \mintinline[breakanywhere,bgcolor=code_bg_pro]{C++}{LONGLONG_MAX}: a constant, as in \mintinline[breakanywhere,bgcolor=code_bg_pro]{C++}{INT_MAX}
    \end{itemize}











