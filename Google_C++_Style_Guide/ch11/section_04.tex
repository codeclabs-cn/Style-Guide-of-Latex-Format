%! Author = codeclabs-cn
%! Date = 2022/12/30

\section{Function Comments}\label{sec:function-comments}
Declaration comments describe use of the function (when it is non-obvious); comments at the definition of a function describe operation.
    \subsection{Function Declarations}
    Almost every function declaration should have comments immediately preceding it that describe what the function does and how to use it. These comments may be omitted only if the function is simple and obvious (e.g., simple accessors for obvious properties of the class). Private methods and functions declared in \mintinline[breakanywhere,bgcolor=code_bg_pro]{C++}{.cc} files are not exempt. Function comments should be written with an implied subject of \emph{This function} and should start with the verb phrase; for example, \enquote{Opens the file}, rather than \enquote{Open the file}. In general, these comments do not describe how the function performs its task. Instead, that should be left to comments in the function definition.

    Types of things to mention in comments at the function declaration:
    \begin{itemize}
        \item What the inputs and outputs are. If function argument names are provided in \enquote{backticks}, then code-indexing tools may be able to present the documentation better.
        \item For class member functions: whether the object remembers reference or pointer arguments beyond the duration of the method call. This is quite common for pointer/reference arguments to constructors.
        \item For each pointer argument, whether it is allowed to be null and what happens if it is.
        \item For each output or input/output argument, what happens to any state that argument is in. (E.g. is the state appended to or overwritten?).
        \item If there are any performance implications of how a function is used.
    \end{itemize}
    Here is an example:
    % \vspace{-\baselineskip}
    \begin{minted}[mathescape,
        linenos,
        numbersep=5pt,
        autogobble, % 左对齐
        breaklines,
        frame=lines,
        framesep=2mm]{C++}
// Returns an iterator for this table, positioned at the first entry
// lexically greater than or equal to `start_word`. If there is no
// such entry, returns a null pointer. The client must not use the
// iterator after the underlying GargantuanTable has been destroyed.
//
// This method is equivalent to:
//    std::unique_ptr<Iterator> iter = table->NewIterator();
//    iter->Seek(start_word);
//    return iter;
std::unique_ptr<Iterator> GetIterator(absl::string_view start_word) const;
    \end{minted}

    However, do not be unnecessarily verbose or state the completely obvious.

    When documenting function overrides, focus on the specifics of the override itself, rather than repeating the comment from the overridden function. In many of these cases, the override needs no additional documentation and thus no comment is required.

    When commenting constructors and destructors, remember that the person reading your code knows what constructors and destructors are for, so comments that just say something like \enquote{destroys this object} are not useful. Document what constructors do with their arguments (for example, if they take ownership of pointers), and what cleanup the destructor does. If this is trivial, just skip the comment. It is quite common for destructors not to have a header comment.

    \subsection{Function Definitions}
    If there is anything tricky about how a function does its job, the function definition should have an explanatory comment. For example, in the definition comment you might describe any coding tricks you use, give an overview of the steps you go through, or explain why you chose to implement the function in the way you did rather than using a viable alternative. For instance, you might mention why it must acquire a lock for the first half of the function but why it is not needed for the second half.

    Note you should \emph{not} just repeat the comments given with the function declaration, in the \mintinline[breakanywhere,bgcolor=code_bg_pro]{C++}{.h} file or wherever. It's okay to recapitulate briefly what the function does, but the focus of the comments should be on how it does it.
