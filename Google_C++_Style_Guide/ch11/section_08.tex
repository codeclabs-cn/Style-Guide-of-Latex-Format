%! Author = codeclabs-cn
%! Date = 2022/12/30

\section{Don'ts}\label{sec:don'ts}
Do not state the obvious. In particular, don't literally describe what code does, unless the behavior is nonobvious to a reader who understands C++ well. Instead, provide higher level comments that describe \emph{why} the code does what it does, or make the code self describing.

Compare this:
% \vspace{-\baselineskip}
\begin{minted}[bgcolor=code_bg_con]{C++}
// Find the element in the vector.  <-- Bad: obvious!
if (std::find(v.begin(), v.end(), element) != v.end()) {
  Process(element);
}
\end{minted}
To this:
% \vspace{-\baselineskip}
\begin{minted}{C++}
// Process "element" unless it was already processed.
if (std::find(v.begin(), v.end(), element) != v.end()) {
  Process(element);
}
\end{minted}
Self-describing code doesn't need a comment. The comment from the example above would be obvious:
% \vspace{-\baselineskip}
\begin{minted}{C++}
if (!IsAlreadyProcessed(element)) {
  Process(element);
}
\end{minted}
