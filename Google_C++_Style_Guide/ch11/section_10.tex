%! Author = codeclabs-cn
%! Date = 2022/12/30

\section{TODO Comments}\label{sec:todo-comments}
Use \mintinline[breakanywhere,bgcolor=code_bg_pro]{C++}{TODO} comments for code that is temporary, a short-term solution, or good-enough but not perfect.

\mintinline[breakanywhere,bgcolor=code_bg_pro]{C++}{TODO}s should include the string \mintinline[breakanywhere,bgcolor=code_bg_pro]{C++}{TODO} in all caps, followed by the name, e-mail address, bug ID, or other identifier of the person or issue with the best context about the problem referenced by the \mintinline[breakanywhere,bgcolor=code_bg_pro]{C++}{TODO}. The main purpose is to have a consistent \mintinline[breakanywhere,bgcolor=code_bg_pro]{C++}{TODO} that can be searched to find out how to get more details upon request. A \mintinline[breakanywhere,bgcolor=code_bg_pro]{C++}{TODO} is not a commitment that the person referenced will fix the problem. Thus when you create a \mintinline[breakanywhere,bgcolor=code_bg_pro]{C++}{TODO} with a name, it is almost always your name that is given.
% \vspace{-\baselineskip}
\begin{minted}{C++}
// TODO(kl@gmail.com): Use a "*" here for concatenation operator.
// TODO(Zeke) change this to use relations.
// TODO(bug 12345): remove the "Last visitors" feature.
\end{minted}
If your \mintinline[breakanywhere,bgcolor=code_bg_pro]{C++}{TODO} is of the form \enquote{At a future date do something} make sure that you either include a very specific date (\enquote{Fix by November 2005}) or a very specific event (\enquote{Remove this code when all clients can handle XML responses.}).
