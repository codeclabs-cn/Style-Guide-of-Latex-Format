%! Author = codeclabs-cn
%! Date = 2022/12/28

\chapter{Comments}\label{ch:comments}

\begin{introduction}
    \item \hyperref[sec:comment-style]{Comment Style}
    \item \hyperref[sec:file-comments]{File Comments}
    \item \hyperref[sec:class-comments]{Class Comments}
    \item \hyperref[sec:function-comments]{Function Comments}
    \item \hyperref[sec:variable-comments]{Variable Comments}
    \item \hyperref[sec:implementation-comments]{Implementation Comments}
    \item \hyperref[sec:function-argument-comments]{Function Argument Comments}
    \item \hyperref[sec:don'ts]{Don'ts}
    \item \hyperref[sec:punctuation-spelling-and-grammar]{Punctuation, Spelling, and Grammar}
    \item \hyperref[sec:todo-comments]{TODO Comments}
\end{introduction}

Comments are absolutely vital to keeping our code readable. The following rules describe what you should comment and where. But remember: while comments are very important, the best code is self-documenting. Giving sensible names to types and variables is much better than using obscure names that you must then explain through comments.

When writing your comments, write for your audience: the next contributor who will need to understand your code. Be generous — the next one may be you!

\subimport{./}{Google_C++_Style_Guide/ch11/section_01.tex}
\subimport{./}{Google_C++_Style_Guide/ch11/section_02.tex}
\subimport{./}{Google_C++_Style_Guide/ch11/section_03.tex}
\subimport{./}{Google_C++_Style_Guide/ch11/section_04.tex}
\subimport{./}{Google_C++_Style_Guide/ch11/section_05.tex}
\subimport{./}{Google_C++_Style_Guide/ch11/section_06.tex}
\subimport{./}{Google_C++_Style_Guide/ch11/section_07.tex}
\subimport{./}{Google_C++_Style_Guide/ch11/section_08.tex}
\subimport{./}{Google_C++_Style_Guide/ch11/section_09.tex}
\subimport{./}{Google_C++_Style_Guide/ch11/section_10.tex}
