%! Author = codeclabs-cn
%! Date = 2022/12/30

\section{Line Length}\label{sec:line-length}
Each line of text in your code should be at most 80 characters long.

We recognize that this rule is controversial, but so much existing code already adheres to it, and we feel that consistency is important.

\subsection{Pros}
Those who favor this rule argue that it is rude to force them to resize their windows and there is no need for anything longer. Some folks are used to having several code windows side-by-side, and thus don't have room to widen their windows in any case. People set up their work environment assuming a particular maximum window width, and 80 columns has been the traditional standard. Why change it?

\subsection{Cons}
Proponents of change argue that a wider line can make code more readable. The 80-column limit is an hidebound throwback to 1960s mainframes; modern equipment has wide screens that can easily show longer lines.

\subsection{Decision}
80 characters is the maximum.

A line may exceed 80 characters if it is
\begin{itemize}
    \item a comment line which is not feasible to split without harming readability, ease of cut and paste or auto-linking -- e.g., if a line contains an example command or a literal URL longer than 80 characters.
    \item a string literal that cannot easily be wrapped at 80 columns. This may be because it contains URIs or other semantically-critical pieces, or because the literal contains an embedded language, or a multiline literal whose newlines are significant like help messages. In these cases, breaking up the literal would reduce readability, searchability, ability to click links, etc. Except for test code, such literals should appear at namespace scope near the top of a file. If a tool like Clang-Format doesn't recognize the unsplittable content, \href{https://clang.llvm.org/docs/ClangFormatStyleOptions.html#disabling-formatting-on-a-piece-of-code}{disable the tool} around the content as necessary.

    (We must balance between usability/searchability of such literals and the readability of the code around them.)
    \item an include statement.
    \item a \hyperref[sec:the-define-guard]{header guard}
    \item a using-declaration
\end{itemize}
