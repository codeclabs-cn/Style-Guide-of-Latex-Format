%! Author = codeclabs-cn
%! Date = 2022/12/30

\section{Non-ASCII Characters}\label{sec:non-ascii-characters}
Non-ASCII characters should be rare, and must use UTF-8 formatting.

You shouldn't hard-code user-facing text in source, even English, so use of non-ASCII characters should be rare. However, in certain cases it is appropriate to include such words in your code. For example, if your code parses data files from foreign sources, it may be appropriate to hard-code the non-ASCII string(s) used in those data files as delimiters. More commonly, unittest code (which does not need to be localized) might contain non-ASCII strings. In such cases, you should use UTF-8, since that is an encoding understood by most tools able to handle more than just ASCII.

Hex encoding is also OK, and encouraged where it enhances readability — for example, \mintinline[breakanywhere,bgcolor=code_bg_pro]{C++}{"\xEF\xBB\xBF"}, or, even more simply, \mintinline[breakanywhere,bgcolor=code_bg_pro]{C++}{u8"\uFEFF"}, is the Unicode zero-width no-break space character, which would be invisible if included in the source as straight UTF-8.

Use the \mintinline[breakanywhere,bgcolor=code_bg_pro]{C++}{u8} prefix to guarantee that a string literal containing \mintinline[breakanywhere,bgcolor=code_bg_pro]{C++}{"\uXXXX"} escape sequences is encoded as UTF-8. Do not use it for strings containing non-ASCII characters encoded as UTF-8, because that will produce incorrect output if the compiler does not interpret the source file as UTF-8.

You shouldn't use \mintinline[breakanywhere,bgcolor=code_bg_pro]{C++}{char16_t} and \mintinline[breakanywhere,bgcolor=code_bg_pro]{C++}{char32_t} character types, since they're for non-UTF-8 text. For similar reasons you also shouldn't use \mintinline[breakanywhere,bgcolor=code_bg_pro]{C++}{wchar_t} (unless you're writing code that interacts with the Windows API, which uses \mintinline[breakanywhere,bgcolor=code_bg_pro]{C++}{wchar_t} extensively).
