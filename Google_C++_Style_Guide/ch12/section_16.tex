%! Author = codeclabs-cn
%! Date = 2022/12/30

\section{Class Format}\label{sec:class-format}
Sections in \mintinline[breakanywhere,bgcolor=code_bg_pro]{C++}{public}, \mintinline[breakanywhere,bgcolor=code_bg_pro]{C++}{protected} and \mintinline[breakanywhere,bgcolor=code_bg_pro]{C++}{private} order, each indented one space.

The basic format for a class definition (lacking the comments, see \hyperref[sec:class-comments]{Class Comments} for a discussion of what comments are needed) is:
% \vspace{-\baselineskip}
\begin{minted}{C++}
class MyClass : public OtherClass {
 public:      // Note the 1 space indent!
  MyClass();  // Regular 2 space indent.
  explicit MyClass(int var);
  ~MyClass() {}

  void SomeFunction();
  void SomeFunctionThatDoesNothing() {
  }

  void set_some_var(int var) { some_var_ = var; }
  int some_var() const { return some_var_; }

 private:
  bool SomeInternalFunction();

  int some_var_;
  int some_other_var_;
};
\end{minted}

Things to note:
\begin{itemize}
    \item Any base class name should be on the same line as the subclass name, subject to the 80-column limit.
    \item The \mintinline[breakanywhere,bgcolor=code_bg_pro]{C++}{public:}, \mintinline[breakanywhere,bgcolor=code_bg_pro]{C++}{protected:}, and \mintinline[breakanywhere,bgcolor=code_bg_pro]{C++}{private:} keywords should be indented one space.
    \item Except for the first instance, these keywords should be preceded by a blank line. This rule is optional in small classes.
    \item Do not leave a blank line after these keywords.
    \item The public section should be first, followed by the \mintinline[breakanywhere,bgcolor=code_bg_pro]{C++}{protected} and finally the \mintinline[breakanywhere,bgcolor=code_bg_pro]{C++}{private} section.
    \item See \hyperref[sec:declaration-order]{Declaration Order} for rules on ordering declarations within each of these sections.
\end{itemize}
